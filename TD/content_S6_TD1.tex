\newcommand{\numTD}{TD1}
\newcommand{\themeTD}{Analyse en caractères}
\newcommand{\file}{toto.tex}

\begin{center}
\begin{tabular}{|p{2cm}p{14cm}|}
\hline
{\includegraphics[width=1.8cm,viewport=0 0 337 248]{../CM/images/sorbonne.png}} & \raisebox{2ex}{\begin{Large}\textbf{Programmation de Modèles Linguistiques (I)}\end{Large}}\\

%2019-2020& \raisebox{2ex}{\begin{large}\textbf{L5SOPROG L3 Sciences du Langage}\end{large}}\\
2023-2024& \raisebox{2ex}{(L5SOPROG L3 Sciences du Langage)}\\
   & \begin{large}\textbf{\numTD}\end{large} \begin{large} \textbf{\themeTD}\end{large} \\
&\\
& Caroline Koudoro-Parfait et Luis Gil Moreno Jimenez, Sorbonne Université \\
\hline
\end{tabular}
\end{center}


\hrule
%%%%%%%%%%%%%%%%%%%%%%%%%EN-TETE%%%%%%%%%%%%%%%%%%%%%%%%%%%%%
%\renewcommand{\contentsname}{Sommaire du TD}
%\tableofcontents
%\newpage

\noindent\fcolorbox{red}{lightgray}{
\begin{minipage}{12cm}
\section*{Objectifs}

\begin{itemize}
 \item Rappels de manipulations simples sur un texte avec l'environnement de travail spyder 
 \item Identifier les propriétés morphologiques d'une langue
 \item Les comparer à d'autres
\end{itemize}
\end{minipage}
}
\newline
\section{Identification de langue améliorée}
~\\
\vspace{-1cm}

\textbf{Données :}
\begin{enumerate}
  \item corpus multilingue vu en cours
\end{enumerate}

\textbf{Énoncé}

\begin{enumerate}
  \item En vous appuyant sur la \textit{baseline} en mots, vue dans le TD4 du semestre 5, travailler avec des tri-grammes de caractères
  \item ré-évaluer ce nouveau modèle
  \item Donner le diagnostic de langue avec un score de confiance en pourcentage
  \item Donner les autres langue possibles du document
\end{enumerate}

\textbf{Attendus}
\begin{enumerate}
  \item Le programme doit être présenté selon \textit{les bonnes pratiques de programmation}
  \item Le programme doit être factorisé
  \item Le programme doit être développé sous l'environnement spyder 
\end{enumerate}

\textbf{Principaux outils nécessaires :}
\begin{itemize}
  \item Json
  \item CountVectorizer/TfIdfVectorizer
\end{itemize}



\section{Reconnaissance d'entités nommées améliorée}
\textbf{Données :}
\begin{enumerate}
  \item Un jeu de données déjà annoté, au format CSV : \textit{CSV\_annote} sur Moodle
 \item des textes a annoter automatiquement 
\end{enumerate}

\textbf{Résultat attendu :}

\begin{enumerate}
  \item Annoter les textes au format IOB avec spaCy
  \item Récupérer les Annotations dans les csv et les mettre au format IOB
  \item Aligner les entités pour calculer la précision, le rappel et le f-score
\end{enumerate}   
  


\textbf{Principaux outils nécessaires :}
\begin{itemize}
  \item csv 
  \item Spacy
  \item Json
  
\end{itemize}



\vspace{0.5cm}
\subsection{Annoter les entités nommées automatiquement avec le format IOB}
\vspace{0.5cm}

\vspace{0.5cm}

\begin{python}
import spacy

nlp = spacy.load(model_langue)
doc = nlp(text)

for ent in doc.ents:
    print(ent.text, ent.start_char, ent.end_char, ent.label_)
\end{python}

Le programme doit permettre de :

\begin{itemize}
    \item  récupérer les entités nommées dans le texte et leur label (PER : personne ; LOC : localisation ; MISC : Divers ; ....)   en utilisant \textit{spaCy }
    \item sauvegarder un dictionnaire en json, en utuilisant le code du TD4. 
 \item d' utiliser deux modèles de langue de \textit{spaCy}. Par exemple : 

    \begin{itemize}
    \item \textit{fr\_core\_news\_sm} 
    \item \textit {fr\_core\_news\_lg} 
\end{itemize}
\item indiquer le temps de travail de l'outil en l'affichant
\end{itemize}

On attend un fichier de sortie pour chaque modèle de langue, structuré ainsi :
%Le dictionnaire attendu est structuré comme suit :

 \begin{python}   
  
 \end{python}
    
   

\begin{center}

\noindent\fcolorbox{blue}{lightgray}{
	\begin{minipage}{15cm}
\section*{Devoir}

\begin{itemize}
 \item 1 script python .py
 \item quelques phrases de conclusion sur les résultats (qu’est-ce qui était attendu, qu’est-ce qui est inattendu ?)
	\item quelques phrases sur l'environnement de développement Spyder comparé à Jupyter notebook, point fort, faible ...

Vous déposerez sur Moodle une archive zip nommée NUMETU.zip (où NUMETU est votre numéro d’étudiant) et contenant :


\item Votre code exporté au format Python .py (et pas ipynb)
\item le PDF du document que vous avez produit
 
 
 Date limite : indiquée sur le Moodle !
\end{itemize}
\end{minipage}
}
	\end{center}

\newcommand{\numTD}{TD1}
\newcommand{\themeTD}{Analyse en caractères}
\newcommand{\file}{toto.tex}

\begin{center}
\begin{tabular}{|p{2cm}p{14cm}|}
\hline
{\includegraphics[width=1.8cm,viewport=0 0 337 248]{../CM/images/sorbonne.png}} & \raisebox{2ex}{\begin{Large}\textbf{Programmation de Modèles Linguistiques (I)}\end{Large}}\\

%2019-2020& \raisebox{2ex}{\begin{large}\textbf{L5SOPROG L3 Sciences du Langage}\end{large}}\\
2023-2024& \raisebox{2ex}{(L5SOPROG L3 Sciences du Langage)}\\
   & \begin{large}\textbf{\numTD}\end{large} \begin{large} \textbf{\themeTD}\end{large} \\
&\\
& Caroline Koudoro-Parfait et Luis Gil Moreno Jimenez, Sorbonne Université \\
\hline
\end{tabular}
\end{center}


\hrule
%%%%%%%%%%%%%%%%%%%%%%%%%EN-TETE%%%%%%%%%%%%%%%%%%%%%%%%%%%%%
%\renewcommand{\contentsname}{Sommaire du TD}
%\tableofcontents
%\newpage

\noindent\fcolorbox{red}{lightgray}{
\begin{minipage}{12cm}
\section*{Objectifs}

\begin{itemize}
 \item Rappels de manipulations simples sur un texte avec l'environnement de travail spyder 
 \item Identifier les propriétés morphologiques d'une langue
 \item Les comparer à d'autres
\end{itemize}
\end{minipage}
}
\newline
\section{Identification de langue améliorée}
~\\
\vspace{-1cm}

\textbf{Données :}
\begin{enumerate}
  \item corpus multilingue vu en cours
\end{enumerate}

\textbf{Énoncé}

\begin{enumerate}
  \item En vous appuyant sur la \textit{baseline} en mots vue dans le TD4 du semestre 5, travailler avec des tri-grammes de caractères
  \item ré-évaluer ce nouveau modèle
  \item Donner le diagnostic de langue avec un score de confiance en pourcentage
  \item Évaluer le taux de réussite du programme en calculant les VP, FP, FN,VN.
  \item Donner les autres langue possibles du document \ding{219} Langues proches
  \item Représenter graphiquement avec Matplotlib\footnote{\url{https://matplotlib.org/}} ou Seaborn\footnote{\url{https://seaborn.pydata.org/}} pour chaque langue quelles sont les langues les plus proches.
\end{enumerate}

\textbf{Attendus}
\begin{enumerate}
  \item Le programme doit être présenté selon \textit{les bonnes pratiques de programmation}
  \item Le programme doit être factorisé
  \item Vous devez choisir des structures de données pertinente pour stocker vos données
  \item Le programme doit être développé sous l'environnement spyder 
\end{enumerate}

\textbf{Principaux outils nécessaires :}
\begin{itemize}
  \item Json
  \item CountVectorizer/TfIdfVectorizer
\end{itemize}





\begin{center}

\noindent\fcolorbox{blue}{lightgray}{
	\begin{minipage}{15cm}
\section*{Devoir}

\begin{itemize}
 \item 1 script python .py
 \item quelques phrases de conclusion sur les résultats (qu’est-ce qui était attendu, qu’est-ce qui est inattendu ?)
	\item quelques phrases sur l'environnement de développement Spyder comparé à Jupyter notebook, point fort, faible ...

Vous déposerez sur Moodle une archive zip nommée NUMETU.zip (où NUMETU est votre numéro d’étudiant) et contenant :


\item Votre code exporté au format Python .py (et pas ipynb)
\item le PDF du document que vous avez produit
 
 
 Date limite : indiquée sur le Moodle !
\end{itemize}
\end{minipage}
}
	\end{center}

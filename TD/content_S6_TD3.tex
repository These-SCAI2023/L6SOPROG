\newcommand{\numTD}{TD3}
\newcommand{\themeTD}{Clustering avec un algorithm d'affinité de propagation}
\newcommand{\file}{toto.tex}

\begin{center}
\begin{tabular}{|p{2cm}p{14cm}|}
\hline
{\includegraphics[width=1.8cm,viewport=0 0 337 248]{../CM/images/sorbonne.png}} & \raisebox{2ex}{\begin{Large}\textbf{Programmation de Modèles Linguistiques (I)}\end{Large}}\\

%2019-2020& \raisebox{2ex}{\begin{large}\textbf{L5SOPROG L3 Sciences du Langage}\end{large}}\\
2023-2024& \raisebox{2ex}{(L5SOPROG L3 Sciences du Langage)}\\
   & \begin{large}\textbf{\numTD}\end{large} \begin{large} \textbf{\themeTD}\end{large} \\
&\\
& Caroline Koudoro-Parfait et Luis Gil Moreno Jimenez, Sorbonne Université \\
\hline
\end{tabular}
\end{center}


\hrule
%%%%%%%%%%%%%%%%%%%%%%%%%EN-TETE%%%%%%%%%%%%%%%%%%%%%%%%%%%%%
%\renewcommand{\contentsname}{Sommaire du TD}
%\tableofcontents
%\newpage

\noindent\fcolorbox{red}{lightgray}{
\begin{minipage}{12cm}
\section*{Objectifs}

\begin{itemize}
 \item Lire des fichiers selon une architecture de dossier données
 \item Utiliser la vectorisation de texte et la distance cosinus
  \item Construire une matrice 
 \item Produire une représentation graphique des résultats
\end{itemize}
\end{minipage}
}
\newline
\section{Clusters}
~\\
\vspace{-1cm}


\textbf{Données :}
\begin{enumerate}
  \item Un jeu de données déjà annoté automatiquement avec spaCy au format IOB2
\end{enumerate}


\textbf{Énoncé}
Vous formerez des clusters à partir des sorties de reconnaissances d'entités nommées, pour représenter les formes les plus proches des entités, \textit{par ex.} pour réunir dans un même cluster \textit{"Morlincourt", "Morlincourtl", "Mlorlincourtl"}.


\textbf{Attendus}
\begin{enumerate}
  \item Le programme doit être présenté selon \textit{les bonnes pratiques de programmation}
  \item Le programme doit être factorisé
  \item Vous devez choisir des structures de données pertinentes pour stocker vos données
  \item Le programme doit être développé sous l'environnement spyder 
\end{enumerate}

\textbf{Principaux outils nécessaires :}
\begin{itemize}
  \item scikit learn
  \item Json
  
\end{itemize}

\vspace{0.5cm}
\subsection{Réflexions et conception en amont}
\vspace{0.5cm}
\label{sec:amont}

Un cluster en traitement automatique des langues naturelles est un groupe de tokens. Pour créer la matrices qui va permettent de former les groupes on peut utiliser :
\begin{itemize}
\item différents types de vectorisation : au grain mot ou au grain caractère (n-gram),
\item différents type de distances permettant de déterminer les tokens les plus proches,
\item plusieurs type d'algorithme pour grouper les tokens les tokens les plus proches.
\end{itemize}

En amont du développement de votre programme vous réfléchirez à quels types de groupes d'entités vous voulez former ?\\
Vous Rédigerez quelques lignes précisant vos idées et proposerez un plan pour développer votre programme.

\subsection{Développement du programme}


\vspace{0.5cm}
\subsubsection*{Lire les fichiers}
\vspace{0.5cm}
Vous utiliserez les fichiers annotés automatiquement avec spaCy au format IOB2 pour préparer une entrée adéquate pour le programme de clusterisation des entités nommées. Le programme attend en entrée un ensemble (set) des tokens.

\begin{python}
Set_00 = set(liste)
Liste_00 = list(Set_00)#on change le type du set en liste pour pouvoir le parcourir
liste_words=[]
        
for l in Liste_00:
                
            if len(l)!=1:
                liste_words.append(l)
        
            words = np.asarray(liste_words)#on créer le tableau préparant la matrice
            for w in words:
                liste_vecteur=[]#on initialise la liste permettant de stocker les vecteurs
            
                    
                for w2 in words:#ici on va utiliser la vectorisation
                
\end{python}               
%\url{}
\vspace{0.5cm}

\subsubsection*{Vectorisation}
\begin{python}   
import sklearn
from sklearn.feature_extraction.text import CountVectorizer

 \end{python}
 
\subsubsection*{Calcul de distances}
 \begin{python}   
import distance
from sklearn.neighbors import DistanceMetric
 \end{python}

\subsubsection*{Construire la matrice}
 \begin{python}   

 \end{python}

\subsection*{Utiliser l'algorithme d'\textit{Affinité de propagation}}
 \begin{python}   

 \end{python}



\subsection*{Stocker les résultats}
 \begin{python}   

 \end{python}
 
\section{Représenter graphiquement les résultats}
 \begin{python}   

 \end{python}

\begin{center}

\noindent\fcolorbox{blue}{lightgray}{
	\begin{minipage}{15cm}
\section*{Devoir}

\begin{itemize}
 \item 1 ou plusieurs script(s) python .py,
 \item 1 PDF présentant :
 \begin{itemize}
 \item La rédaction et le plan attendus en partie \ref{sec:amont}
 \item quelques phrases de conclusion sur les résultats (qu’est-ce qui était attendu, qu’est-ce qui est inattendu ?)
\end{itemize}

Vous déposerez sur Moodle une archive zip nommée NUMETU.zip (où NUMETU est votre numéro d’étudiant) et contenant :


\item Votre code exporté au format Python .py (et pas ipynb)
\item le PDF du document que vous avez produit
 
 
 Date limite : indiquée sur le Moodle !
\end{itemize}
\end{minipage}
}
	\end{center}

\newcommand{\numTD}{TD3}
\newcommand{\themeTD}{Clustering avec un algorithm d'affinité de propagation}
\newcommand{\file}{toto.tex}

\begin{center}
\begin{tabular}{|p{2cm}p{14cm}|}
\hline
{\includegraphics[width=1.8cm,viewport=0 0 337 248]{../CM/images/sorbonne.png}} & \raisebox{2ex}{\begin{Large}\textbf{Programmation de Modèles Linguistiques (I)}\end{Large}}\\

%2019-2020& \raisebox{2ex}{\begin{large}\textbf{L5SOPROG L3 Sciences du Langage}\end{large}}\\
2023-2024& \raisebox{2ex}{(L5SOPROG L3 Sciences du Langage)}\\
   & \begin{large}\textbf{\numTD}\end{large} \begin{large} \textbf{\themeTD}\end{large} \\
&\\
& Caroline Koudoro-Parfait et Luis Gil Moreno Jimenez, Sorbonne Université \\
\hline
\end{tabular}
\end{center}


\hrule
%%%%%%%%%%%%%%%%%%%%%%%%%EN-TETE%%%%%%%%%%%%%%%%%%%%%%%%%%%%%
%\renewcommand{\contentsname}{Sommaire du TD}
%\tableofcontents
%\newpage

\noindent\fcolorbox{red}{lightgray}{
\begin{minipage}{12cm}
\section*{Objectifs}

\begin{itemize}
 \item Lire des fichiers selon une architecture de dossier données
 \item Utiliser la vectorisation de texte et la distance cosinus
  \item Construire une matrice 
 \item Produire une représentation graphique des résultats
\end{itemize}
\end{minipage}
}
\newline
\section{Clusters}
~\\
\vspace{-1cm}


\textbf{Données :}
\begin{enumerate}
  \item Un jeu de données déjà annoté automatiquement avec spaCy
\end{enumerate}


\textbf{Énoncé}

\begin{enumerate}
  \item 
\end{enumerate}

\textbf{Attendus}
\begin{enumerate}
  \item Le programme doit être présenté selon \textit{les bonnes pratiques de programmation}
  \item Le programme doit être factorisé
  \item Vous devez choisir des structures de données pertinente pour stocker vos données
  \item Le programme doit être développé sous l'environnement spyder 
\end{enumerate}

\textbf{Principaux outils nécessaires :}
\begin{itemize}
  \item scikit learn
  \item Json
  
\end{itemize}



\vspace{0.5cm}
\subsection*{Lire les fichiers}
\vspace{0.5cm}
Vous utiliserez les fichiers annotés automatiquement avec spaCy pour préparer une entrée adéquate pour le programme de clusterisation des entités nommées.
%\url{}
\vspace{0.5cm}

\subsection*{Vectorisation et calcul de distances}
 \begin{python}   

 \end{python}

\subsection*{Construire la matrice}
 \begin{python}   

 \end{python}

\subsection*{Utiliser l'algorithme d'\textit{Affinité de propagation}}
 \begin{python}   

 \end{python}



\subsection*{Stocker les résultats}
 \begin{python}   

 \end{python}
 
 \subsection*{Représenter graphiquement les résultats}
 \begin{python}   

 \end{python}

\begin{center}

\noindent\fcolorbox{blue}{lightgray}{
	\begin{minipage}{15cm}
\section*{Devoir}

\begin{itemize}
 \item 1 script python .py
 \item quelques phrases de conclusion sur les résultats (qu’est-ce qui était attendu, qu’est-ce qui est inattendu ?)


Vous déposerez sur Moodle une archive zip nommée NUMETU.zip (où NUMETU est votre numéro d’étudiant) et contenant :


\item Votre code exporté au format Python .py (et pas ipynb)
\item le PDF du document que vous avez produit
 
 
 Date limite : indiquée sur le Moodle !
\end{itemize}
\end{minipage}
}
	\end{center}

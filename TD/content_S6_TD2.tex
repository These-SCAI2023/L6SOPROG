\newcommand{\numTD}{TD1}
\newcommand{\themeTD}{Analyse en caractères}
\newcommand{\file}{toto.tex}

\begin{center}
\begin{tabular}{|p{2cm}p{14cm}|}
\hline
{\includegraphics[width=1.8cm,viewport=0 0 337 248]{../CM/images/sorbonne.png}} & \raisebox{2ex}{\begin{Large}\textbf{Programmation de Modèles Linguistiques (I)}\end{Large}}\\

%2019-2020& \raisebox{2ex}{\begin{large}\textbf{L5SOPROG L3 Sciences du Langage}\end{large}}\\
2023-2024& \raisebox{2ex}{(L5SOPROG L3 Sciences du Langage)}\\
   & \begin{large}\textbf{\numTD}\end{large} \begin{large} \textbf{\themeTD}\end{large} \\
&\\
& Caroline Koudoro-Parfait et Luis Gil Moreno Jimenez, Sorbonne Université \\
\hline
\end{tabular}
\end{center}


\hrule
%%%%%%%%%%%%%%%%%%%%%%%%%EN-TETE%%%%%%%%%%%%%%%%%%%%%%%%%%%%%
%\renewcommand{\contentsname}{Sommaire du TD}
%\tableofcontents
%\newpage

\noindent\fcolorbox{red}{lightgray}{
\begin{minipage}{12cm}
\section*{Objectifs}

\begin{itemize}
\item Analyses linguistiques appliquées à un texte: Lemmatisation, Tokenization, POS tagging.
  \item Organiser son/ses scripts de manière lisible
  \item 
 \item 
  \item représenter graphiquement des résultats
\end{itemize}
\end{minipage}
}
\newline


\textbf{Principaux outils nécessaires :}
\begin{itemize}
  \item spaCy
  \item Matplotlib/ Seaborn
  \item Json
  
\end{itemize}

\section{Utiliser différents outils de spaCy pour l'analyse linguistique}
Utiliser les outils proposés par spaCy pour effectuer les traitements suivants :
\begin{enumerate}
\item Lemmatisation
\item POS tagging
\end{enumerate}
\ding{220} Vous stockerez les résultats dans la structure de données et dans un fichier dont l'extension vous semble les plus appropriées.\\
 
\ding{220} Vous proposerez une représentation graphique qui vous semble pertinente

\vspace{0.5cm}
\section{Différentes tokenisations}
\vspace{0.5cm}

\begin{description}
\item [étape 1] Tokeniser le texte en utilisant la fonction native python split()
  \item [étape 2] Tokeniser les textes en utilisant spaCy
 \item [étape 3] stocker les résultats dans une/des structures de données qui vous semblent appropriées
 \item [étape 4] Exprimer les résultats sous forme d'un graphique représentant une loi de Zipf. Ce graphique permet de comparer la courbe de la loi de Zipf pour les résultats avec la tokenisation split() et la tokenisation de spaCy. Quelles observations pouvez-vous formuler ?
\end{description}



\vspace{0.5cm}

\begin{python}

\end{python}



 \begin{python}   
  
 \end{python}
    
   

\begin{center}

\noindent\fcolorbox{blue}{lightgray}{
	\begin{minipage}{15cm}
\section*{Devoir}

\begin{itemize}
 \item 1 script python .py
 \item quelques phrases de conclusion sur les résultats (qu’est-ce qui était attendu, qu’est-ce qui est inattendu ?)
	

Vous déposerez sur Moodle une archive zip nommée NUMETU.zip (où NUMETU est votre numéro d’étudiant) et contenant :


\item Votre code exporté au format Python .py (et pas ipynb)
\item le PDF du document que vous avez produit
 
 
 Date limite : indiquée sur le Moodle !
\end{itemize}
\end{minipage}
}
	\end{center}

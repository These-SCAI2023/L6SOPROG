 Pour améliorer nous allons construire des \textbf{fonctions} pour \textbf{factoriser} les traitements et constituer une \textbf{chaîne de traitement} fiable.

\textbf{Etape 1: lire}

Ce qui va changer ici c'est qu'on veut traiter plusieurs textes facilement. Bien sûr on pourrait faire :

\begin{python}
with open("13846-0.txt") as f:#Discours de la Methode
  chaine1 = f.read()
with open("4300-0.txt") as f:#Ulysses
  chaine2 = f.read()
\end{python}
 La première chose que l'on remarque c'est que sur Windows (le problème ne se pose pas avec Mac ou Linux) on arrive à ouvrir le texte en anglais mais pas celui en français. C'est un problème d'encodage des caractères (\texttt{charmap} error). L'encodage c'est la manière dont on stocke les caractères en les changeant en "0" et en "1".
 Pour faire simple, quand on a des caractères accentués on a besoin d'un encodage (\textit{encoding}) adapté. Ici c'est "utf-8", on va faire la modification suivante :
\begin{python}
with open("13846-0.txt", encoding='utf-8') as f:#Discours de la Methode
  chaine1 = f.read()

with open("4300-0.txt") as f:#Ulysses
  chaine2 = f.read()
\end{python}

 On voit que l'on a répété deux fois le même code et aussi que si l'on a des correctiosn à faire, il faudra le faire deux fois.

 Si on a 100 textes à traiter il faut 300 lignes de code, pas très pratique.Nous allons voir avec une fonction ça marche mieux. Pour la fabriquer il faut décomposer notre problème, voir \textbf{ce qui est constant} (factorisable, les opérations) et \textbf{ce qui est variable} (non factorisable, paramètre ou sortie de la fonction).
 On se rend vite compte que ce que l'on veut c'est à partir d'un chemin de fichier (\textbf{input} ou entrant) avoir son contenu sous forme de chaîne de caractères (\textbf{output} ou sortant). Ce qui nous donne :

\begin{python}
def lire_fichier(chemin):
  with open(chemin, encoding = 'utf-8') as f:
    chaine = f.read()
  return chaine

chaine1 = lire_fichier("13846-0.txt")#Discours de la Methode
chaine2 = lire_fichier("4300-0.txt")#Ulysses

\end{python}

\textbf{Etape 2 : découper}

Toujours réfléchir en terme d'entrant/sortant : Quel est l'entrant et le  sortant qu'il faut ajouter dans le squelette ci-contre à la place des XXX, YYY et ZZZ?

\begin{python}
def decouper_en_mots(XXX):
  #on decoupe
  liste_mots = YYY
  return ZZZ

\end{python}

(à vous de réfléchir, réponse page suivante)
\newpage

\begin{python}
def decouper_en_mots(chaine):
  #on decoupe
  liste_mots = chaine.split()
  return liste_mots

liste_mots1 = decouper_en_mots(chaine1)
liste_mots2 = decouper_en_mots(chaine2)
\end{python}

\textbf{Etape 3 : compter}
	
En entrée : la liste de mots

En sortie : les effectifs

NB: vous pouvez mettre des \textit{print} quand vous testez pour bien voir ce qu'il se passe.

\begin{python}

def get_effectifs(liste_mots):
  dic_longueurs = {}
  for mot in liste_mots:
    longueur = len(mot)#la longueur du mot
    if longueur not in dic_longueurs: #on a jamais vu cette longueur de mot
      dic_longueurs[longueur]=1 #
    else: #on a vu cette longueur de mot
      dic_longueurs[longueur]+=1
  return dic_longueurs

\end{python}

Et cette fonction on va l'utiliser directement dans l'étape 5

\textbf{Etape 4 : observer (obsolète)}

 C'était une étape de vérification devenue inutile puisqu'on n'a pas changé les opérations effectuées. 

\textbf{Etape 5 : représenter}

Ici on va pouvoir afficher les deux courbes sur la même figure et cerise sur le gâteau on va la sauvegarder.

\begin{python}
import matplotlib.pyplot as pyplot #import avec alias

for liste in [liste_mots1, liste_mots2]: #on a une liste de liste pour factoriser
  dic_longueurs = get_effectifs(liste)
  liste_effectifs = []
  for toto in range(30):
    if toto in dic_longueurs:#on a donc vu des mots de cette longueur
      liste_effectifs.append(dic_longueurs[toto])
    else:#on en n'a pas vu de cette longueur, on ajoute donc un 0
      liste_effectifs.append(0)
  pyplot.plot(liste_effectifs)#on "dessine" mais dans la boucle

pyplot.show()#"on affiche" mais hors de la boucle (pour avoir tout)

\end{python}

\begin{figure}
\begin{center}
\includegraphics[width=.5\textwidth]{images/TD1_effectifs_total.png}
\caption{"Discours de la Méthode" et "Ulysses" : nombre de mots par longueur (en abscisse), en ordonnée l'effectif}
\end{center}
\end{figure}

On se rend compte que la figure est difficile à interpréter, en effet on travaille en valeur absolue alors que les textes sont de taille différente. On va donc utiliser la taille de chaque texte en mots (avec la fonction \texttt{len}) pour  avoir cette fois une figure avec la proportion de mots de chaque longueur :

\newpage

\begin{python}
#On remplace la ligne :
      liste_effectifs.append(dic_longueurs[toto])
#Par:
      liste_effectifs.append(dic_longueurs[toto]/len(liste))
\end{python}

\begin{figure}
\begin{center}
\includegraphics[width=.5\textwidth]{images/TD1_frequences_total.png}
\caption{"Discours de la Méthode" et "Ulysses" : nombre de mots par longueur (en abscisse), en ordonnée la fréquence}
\end{center}
\end{figure}

Voir page suivante pour un bilan

\newpage
\textbf{Après une dernière étape de factorisation voici où nous en sommes :}

\begin{python}
def lire_fichier(chemin):
  with open(chemin, encoding = "utf-8") as f:
    chaine = f.read()
  return chaine

def decouper_en_mots(chaine):
  liste_mots = chaine.split()
  return liste_mots

def get_effectifs(liste_mots):
  dic_longueurs = {}
  for mot in liste_mots:
    longueur = len(mot)
    if longueur not in dic_longueurs:
      dic_longueurs[longueur]=1 
    else:
      dic_longueurs[longueur]+=1
  return dic_longueurs

def vecteur_longueurs(dic_longueurs, liste_mots):
  liste_effectifs = []
  for toto in range(30):
    if toto in dic_longueurs:
      liste_effectifs.append(dic_longueurs[toto]/len(liste_mots))
    else:
      liste_effectifs.append(0)
  return liste_effectifs

import matplotlib.pyplot as pyplot

for chemin in ["13846-0.txt", "4300-0.txt"]:
  chaine = lire_fichier(chemin)
  liste_mots = decouper_en_mots(chaine)
  dic_longueurs = get_effectifs(liste_mots)
  liste_effectifs = vecteur_longueurs(dic_longueurs, liste_mots)
  pyplot.plot(liste_effectifs)

pyplot.savefig("frequences.png")#le bonus: on sauvegarde
pyplot.show()


\end{python}

C'est pas mal, au prochain TD on améliorera le rendu de la figure (légende, échelle) et on travaillera sur plus de langues.

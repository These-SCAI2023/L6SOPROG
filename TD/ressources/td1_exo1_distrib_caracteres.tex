
 Nous allons implanter le calcul de la distribution des mots d'un texte (son vocabulaire) en fonction de leur taille en caractères.
 Ainsi, vous devrez calculer le nombre de mots uniques pour une taille donnée comme le montre la Figure \ref{distrib}.

\begin{figure}[h]
  \includegraphics[trim = {0cm 1.5cm 0cm 0cm}, clip, width=.6\textwidth]{images/distrib.png}
  \caption{Distribution des mots par rapport à leur taille en caractères dans différentes langues\label{distrib}}
\end{figure}


 L'axe des abscisses représente la taille d'un mot en caractères et l'axe des ordonnées représente le nombre de mots uniques correspondant à cette taille.

Pour préparer votre espace de travail sur votre notebook  \textsc{Jupyter}, créez un dossier TD1 dans lequel vous enregistrerez votre code et vos données.

 Pous constituer un corpus, vous utiliserez les fichiers textes suivants (choisissez plain text utf-8):
\begin{itemize}
\item "Le discours de la méthode" (fr) \url{http://www.gutenberg.org/ebooks/13846} %. Pour que les résultats soient plus significatifs, pensez à supprimer les premières phrases du fichier écrites en anglais.
\item "Ulysses" (en) \url{http://www.gutenberg.org/ebooks/4300}
\end{itemize}

\newpage

\textbf{Les étapes :}

\begin{enumerate}
  \item  \textbf{lire} les textes
  \item  \textbf{découper} en mots (ou tokeniser)
  \item  \textbf{compter} le nombre de mots par taille de caractères
  \item  \textbf{observer} les résultats chiffrés
  \item  \textbf{représenter} cela sur une courbe
\end{enumerate}


\textbf{Etape 1 : lire}

On ouvre le fichier en indiquant son chemin (\textit{path}), si vous avez bien enregistré votre fichier au même endroit que votre code, le nom du fichier suffit
.

\textbf{Si ça ne marche pas c'est que} :
\begin{itemize}
  \item Tout n'est pas au bon endroit (\texttt{File not Found}), regardez dans l'onglet \texttt{files} de \textsc{Jupyter} pour voir où vous êtes.
  \item ou que on a un problème d'encoding (\texttt{charmap}), il faut ajouter ecoding ='utf-8' dans le open : open("13846-0.txt", encoding ="utf-8")
\end{itemize}

Nous allons commencer par le "Discours de la Méthode", si vous avez conservé le nom d'origine il devrait s'appeler "13846-0.txt".

\begin{python}
with open("13846-0.txt") as f:
  chaine = f.read()
\end{python}

Et on affiche un bout du texte pour vérifier que ça marche :

\begin{python}
print(chaine[:100])
\end{python}


\textbf{Etape 2 : découper}

On va très simplement découper en mots avec la \textbf{méthode} \textit{split}
\begin{python}
liste_mots = chaine.split()#approximation des occurrences
print("Nombre de mots : %i" %len(liste_mots))
\end{python}


\textbf{Etape 3 : compter}

On va utiliser un \textbf{dictionnaire} (ou tableau associatif) où l'on va stocker pour chaque longueur en caractères le nombre de mots qu'on a rencontré. Le fonctionnement est le suivant:
\begin{itemize}
  \item pour chaque mot de la liste de mots, on calcule sa longueur
  \item on vérifie si on a déjà rencontré un mot de cette longueur:
  \begin{itemize}
    \item Si c'est le premier mot pour cette longueur on crée une \textbf{clé} pour cette longueur à laquelle on affecte la \textbf{valeur} 1
    \item Sinon, on \textbf{incrémente} de 1 la valeur existante
  \end{itemize}
\end{itemize}

\begin{python}
dic_longueurs = {} #un dictionnaire vide

for mot in liste_mots:
  longueur = len(mot)#la longueur du mot
  if longueur not in dic_longueurs: #on a jamais vu cette longueur de mot
    dic_longueurs[longueur]=1 #
  else: #on a vu cette longueur de mot
    dic_longueurs[longueur]+=1

print(dic_longueurs)#pour avoir une vue de ce qu'on a fait

\end{python}

NB: si le processus ne vous semble pas clair, ajoutez au début de la boucle \textit{for} deux lignes (avec l'indentation) pour suivre le processus pas à pas :

\begin{python}
  print(dic_longueurs)
  dd=input("Appuyez sur Enter pour passer a la suite")
\end{python}

\textbf{Etape 4: observer}

Un dictionnaire n'est pas une structure de données ordonnée, pour vérifier que'on trouve des résultats proche de l'attendu, on va afficher le nombre d'occurences enregistré dans \texttt{dic\_longueurs} pour toutes les longueurs de 1 à 30 en utilisant \textbf{l'itérateur} \textit{range}. Dans le \textit{print} on utilise du \textbf{formatage de chaînes de caractères}\footnote{Voir par exemple \url{https://stackoverflow.com/questions/5082452/string-formatting-vs-format}}.

\begin{python}
for toto in range(1, 31):#de 1 à 30 (31 est exclu)
  nbr_occurences = dic_longueurs[toto]
  print("%i : %i"%(toto, nbr_occurences))
\end{python}

Vous verrez que le code plante car on a des longueurs qui ne sont pas dans le dictionnaire, on va donc améliorer le code de la façon suivante:

\begin{python}
for toto in range(30):
  if toto in dic_longueurs:
    nbr_occurences = dic_longueurs[toto]
    print("%i : %i"%(toto, nbr_occurences))
  else:
    nbr_occurences = 0 
    print("%i : %i"%(toto, nbr_occurences))
\end{python}


\textbf{Etape 5 : représenter}

Et maintenant c'est magique, on va créer une courbe grâce à la librairie \texttt{matplotlib}. On va importer cette librairie et la renommer pour que ça soit plus court à écrire. Puis pour avoir les valeurs à mettre sur la courbe on va lire les valeurs dans l'ordre croissant pour les ranger dans une liste nommée \textit{liste\_effectifs}. Pyplot prend entrée un \textbf{vecteur}, une liste de valeurs ordonnées.

\begin{python}
import matplotlib.pyplot as pyplot #import avec alias

liste_effectifs = []
for toto in range(30):
  if toto in dic_longueurs:#on a donc vu des mots de cette longueur
    liste_effectifs.append(dic_longueurs[toto])
  else:#on en n'a pas vu de cette longueur, on ajoute donc un 0
    liste_effectifs.append(0)
pyplot.plot(liste_effectifs)#on "dessine"
pyplot.show()#"on affiche"

\end{python}

\begin{figure}
\begin{center}
\includegraphics[width=.5\textwidth]{images/TD1_effectifs1.png}
\caption{"Discours de la Méthode" : nombre de mots par longueur (en abscisse), en ordonnée l'effectif}
\end{center}
\end{figure}


Maintenant si on veut faire le même calcul pour l'autre texte on a juste à changer le nom du fichier dans l'étape 1 et à relancer toutes les cellules. Mais si on avait 100 textes à faire ça ne serait pas très pratique. Nous allons donc voir dans l'exercice suivant comment améliorer le code.

Maintenant, nous avons tous les éléments pour réaliser une prédiction selon Bayes.

\subsection{Construction de la classe \textit{classifier}}

\begin{enumerate}
	\item Créez la classe \textit{classifier} avec son constructeur, le constructeur doit recevoir comme arguments les listes : \textit{logAs, Logpis, logpriors}. Notez que les noms sont en pluriels, ca veut dire que ces objets vont recevoir des listes de listes, une par classe.
	\begin{itemize}
		\item le constructeur devra initialiser des variables locales en utilisant les valeur reçues en argument.
	\end{itemize}
	\item Créez la fonction \textit{compute\_log\_likelihood()} qui reçoit trois arguments: \textit{input\_, class\_, self}.
	\begin{itemize}
		\item \textit{input\_} : Contiendra le texte à analyser déjà transformé en valeurs entières.
		\item \textit{class\_} : Contiendra l'étiquette de la classe avec la quelle le texte sera comparé (0 = Poe, 1 = Frost)
		\item \textit{self} : Pour accéder aux objets de la classe.
		\item La fonction doit renvoyer la somme des probabilités d'apparition des bigrammes et des mots en début des phrases trouvés dans \textit{input\_}, selon l'équation suivante : $\log p(s_1...T)=\log \pi_{s1}+\sum_{t=2}^{T}\log A_{s_{t-1},s_t}$.
		\item \textbf{Attention !} le calcul doit être effectué selon la classe donnée comme argument.
	\end{itemize}
	\item Créez la fonction \textit{predict()} avec deux arguments comme entrée : \textit{inputs} et \textit{self}.
	\begin{itemize}
		\item \textit{input} : Contiendra les textes à analyser déjà transformés en valeurs entières.
		\item Si l'on part de l'idée que \textit{inputs} contient un poème par ligne
		\begin{itemize}
			\item La fonction doit renvoyer un vecteur de taille = $len(inputs)$.
			\item Chaque position du vecteur doit contenir la classe prédite pour chaque poème donné en argument en respectant l'ordre d'apparition des poèmes.
			\item N'oubliez pas que la classe prédite doit être celle avec la probabilité plus élevée. Rappelez l'équation suivante :\\
			$k^* = \arg max_k \log p(poem | author = k) p(author = k)$
		\end{itemize}
	\end{itemize}
\end{enumerate}
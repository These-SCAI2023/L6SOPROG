\global\long\def\numTD{TD5}%
 
\global\long\def\themeTD{Entraînement des modèles linguistiques}%
 
\global\long\def\file{toto.tex}%

\begin{center}
\begin{tabular}{|p{2cm}p{14cm}|}
\hline 
{\includegraphics[width=1.8cm,viewport=0 0 337 248]{img/sorbonne}}  & \raisebox{2ex}{\begin{Large}\textbf{Programmation de Modèles Linguistiques
(II)}\end{Large}}\tabularnewline
2023-2024 & \raisebox{2ex}{(L6SOPROG L3 Sciences du Langage)}\tabularnewline
 & \begin{large}\textbf{TD5}\end{large} \begin{large} \textbf{\themeTD}\end{large} \tabularnewline
 & \tabularnewline
 & Luis MORENO et Caroline Koudoro-Parfait, Sorbonne Université \tabularnewline
\hline 
\end{tabular}
\par\end{center}


\hrule %%%%%%%%%%%%%%%%%%%%%%%%%EN-TETE%%%%%%%%%%%%%%%%%%%%%%%%%%%%%
%\renewcommand{\contentsname}{Sommaire du TD}
%\tableofcontents
%\newpage

\noindent %
\fcolorbox{red}{lightgray}{ %
\begin{minipage}[c]{16.5cm}%
 

\section*{Objectifs}
\begin{itemize}
	\item Mettre en pratique les connaissances acquisse en Python pour l'analyse du corpus.
	\item Identifier et extraire les caractéristiques statistiques d'un corpus linguistique. 
	\item Entraîner un modèle de langues à partir d'une approche stochastique.
	\item Concevoir un algorithme génératif capable d'écrire un nouveau texte d'une longitude variable.
\end{itemize}
%
\end{minipage}} \\

\section{Exercice 1 : Préparation des données et entraînement}

\vspace{0.5cm}
Pour notre générateur de texte nous allons procéder à l’analyse d'un corpus. Ce corpus, connu aussi comme corpus d'entraînement sera représentée par les ouvres de \textit{Robert Frost}.

\textbf{MAIS} avant de passer à l'entraînement, nous devons préparer les données afin qu'ils soient traitables pour la machine.

\textbf{Ressources : }

Pour cette implémentation, nous allons avoir besoin de quelques bibliothèques importantes. Considérez que ceci est juste un exemple, si vous avez une autre idée de comment aborder le problème, n’hésitez pas a le faire à votre manière.

Nous allons donc charger les bibliothèques suivantes :

\begin{python}
import numpy as np
import string

np.random.seed(1234)
\end{python}

Si vous essayez de charger ces bibliothèques en local, il ne faut pas oublier de les avoir installées préalablement.

\vspace{0.5cm}

\subsection{Étape 1 : Téléchargement}

La première étape consiste a télécharger le fichier qui contient les textes qui serviront pour entraîner notre modèle. Vous pouvez trouver le fichier dans la section 'Ressources'.

Pour être certains que vous avez bien téléchargé les fichiers et qu'ils sont dans le répertoire principal testez la commande :

\begin{python} 
#Afficher les 10 premières lignes
!head robert_frost.txt 
\end{python}

Observez ce qui se passe...

\subsection{Étape 2 : Prétraitement de données}

\subsubsection{Déclaration d'objets pour le traitement des données}

\begin{itemize}
	\item Créez trois dictionnaires :
	\begin{enumerate}
		\item \textit{initial}
		\item \textit{first\_order}
		\item \textit{second\_order}
	\end{enumerate}
	\item Créez la fonction \textit{remove\_punctuation()} : 
	\begin{itemize}
		\item Elle reçoit par argument le texte à traiter,
		\item et renvoie le texte libre de signes de ponctuation.
	\end{itemize}
	\item Créez la fonction \textit{add2dict()} :
	\begin{itemize}
		\item Cette fonction reçoit comme arguments : 
		\begin{itemize}
			\item un dictionnaire (\textbf{d}) qui sera manipulé en fonction des autres deux arguments
			\item un objet (\textbf{k}), le mot clé qu'on doit chercher dans le dictionnaire,
			\item une objet (\textbf{v}), qui correspond à la valeur associée à \textbf{k}.
		\end{itemize}
		\item La fonction doit renvoyer le dictionnaire \textbf{d}, soit avec un nouvel élément -> $d[k] = [v]$; soit avec la valeur \textbf{v} rajoutée à l'élément existant \textbf{k} -> $d[k].append(v)$
	
	\end{itemize}
\end{itemize}



\subsubsection{Étape 3 : Lecture du corpus}

Nous devons créer une stratégie afin d'extraire tous les bigrammes du texte traité. Commencez par lire le contenu de notre corpus comme dans le code suivant :

\begin{python}
#Initialization d'une boucle pour lire le corpus phrase par phrase
for line in open('robert_frost.txt'):
 #On transforme tous les mots en minuscule
 tokens = remove_punctuation(line.rstrip().lower()).split()
 
\end{python}

Complétez le code en suivant les instructions :

À l’intérieur de la boucle, définissez une deuxième boucle imbriquée afin de lire mot par mot la liste \textit{tokens} et considérez quatre situations :
\begin{enumerate}
	\item Si le mot est en début de phrase, rajoutez le au dictionnaire \textit{initial}. Pensez qu'à la fin le dictionnaire doit contenir les mots en \textit{key} et en \textit{value} leur nb d'occurrences.
	\item Si le mot est au final de la phrase, vous devez utiliser la fonction \textit{add2dict} en envoyant comme argument :
	\begin{enumerate}
		\item le dictionnaire à affecter : \textit{second\_order},
		\item l'élément qui sera intégré sur l'index \textit{key} : bigramme (mot précédent(-1) + mot actuel). Notez qu'au lieu de contenir qu'un mot, ce dictionnaire contiendra des bigramme de mots.
		\item la valeur (\textit{value}) associée au bigramme indiqué précédemment : utilisez la valeur constante 'END', comme ça on saura que ce bigramme constitue la fin de la phrase.
	\end{enumerate}
	\item Si le mot est le second mot de la phrase, utilisez aussi la fonction \textit{add2dict} en envoyant comme argument :
	\begin{enumerate}
		\item Dictionnaire : \textit{first\_order},
		\item \textit{key} : le mot actuel
		\item \textit{value} : comme référence envoyez le mot qui précède le mot actuel (-1).
	\end{enumerate}
	\item Pour les mots qui ne se trouvent pas dans les cas antérieures, utilisez encore la fonction \textit{add2dict} avec les arguments :
	\begin{enumerate}
		\item Dictionnaire : \textit{second\_order},
		\item \textit{key} : bigramme (mot précédent(+2) + mot précédent(-1)), le dictionnaire contiendra aussi des bigrammes des mots qui précèdent le mot actuel.
		\item \textit{value} : comme valeur, envoyez le mot actuel.
	\end{enumerate}
\end{enumerate}

\textbf{Exercice :}
Décrivez l'utilité des éléments que l'on vient de créer ainsi que de leur structures.

\section{Exercice 2 : Normalisation de données et implémentation}

Une fois que les caractéristiques extraites du corpus sont stockées dans nos dictionnaires, nous procédons à l'étape de normalisation.

\subsection{Étape 1 : Dictionnaire \textit{initial}}

À l'aide d'une boucle, parcourez les éléments du dictionnaire \textit{initial}.
Rappelez que dans l'index \textit{key} nous avons les mots qui commencent les phrases analysées, et dans \textit{value} nous avons leur nb d'occurrences, vous devez donc modifier cette dernière valeur en lui divisant par la somme de toutes les occurrences de tous les mots. Inspirez vous de l'exemple suivant :

\begin{python}
sum_occ = sum(initial.values())
for element in initial:
 element.value = element.value / sum_occ
\end{python}

\subsection{Étape 2 : Dictionnaires \textit{first\_order} et \textit{second\_order}}

\subsubsection{fonction \textit{list2pdict()}}
Créez la fonction \textit{list2pdict}, elle recevra comme argument une liste de mots non uniques. La fonction doit renvoyer un dictionnaire avec les mots uniques de la liste placés sur l'index \textit{key}, et en \textit{value} leur nb d'occurrences \textbf{normalisés} (réutilisez la méthode de normalisation appliquée pour le premier dictionnaire).
\begin{python}
def list2pdic(listeMots):
 dictionnaire ={}
 n = len(listeMots)
 for mot in listeMots:
  dictionnaire.key = mot
  dictionnaire.value = dictionnaire.value + 1
  
 for element in dictionnaire:
  element.value = element.value / n

\end{python}

\subsubsection{Implémentation}
Regardez le code suivante et décrivez sa fonctionnalité :
\begin{python}
for t_1, ts in first_order.items():
 # replace list with dictionary of probabilities
 first_order[t_1] = list2pdict(ts)
\end{python}
\vspace{5cm}
Réutilisez le code précédent afin de normaliser aussi le dictionnaire \textit{second\_order}.

\subsection{Fonction \textit{sample\_word()}}
Créez la fonction \textit{sample\_word()} qui devra recevoir comme argument un sous dictionnaire. Prenez le code suivant comme base et complétez la fonction pour qu'elle renvoie le mot avec une valeur cumulée supérieure à $p0$.

\begin{python}
def sample_word(d):
 #Juste une valeur de référence random
 p0 = np.random.random()
 for element in d:
  if element.value.cumulées > p0:
   return element.key #Le mot
\end{python}

\subsection{Fonction \textit{generate()}}
Créez la fonction \textit{generate()} qui devra recevoir comme argument le nombre de lignes qui seront générées. À l’intérieure de la fonction procédez de la manière suivante :

\begin{enumerate}
	\item Déterminez le premier mot du nouveau texte.
	\begin{itemize}
		\item Exécutez la fonction \textit{sample\_word()}
		\item Envoyez comme argument le dictionnaire \textit{initial}
	\end{itemize}
	 
	\item Déterminez le deuxième mot en exécutant la fonction \textit{sample\_word()}. 
	\begin{itemize}
		\item Envoyez comme argument le dictionnaire \textit{first\_order}.
		\item ATTENTION ! : Envoyez pas tout le dictionnaire, rappelez que \textit{first\_order} est un dictionnaire de dictionnaires ! Alors, envoyez QUE le dictionnaire directement associé au mot précédemment déterminé.
	\end{itemize}
	\item À l'aide d'une boucle, déterminez le reste des mots en exécutant la fonction \textit{sample\_word()}. Envoyez comme argument le dictionnaire \textit{second\_order}.
	\begin{itemize}
		\item ATTENTION ! : Envoyez pas tout le dictionnaire, rappelez que \textit{second\_order} est un dictionnaire de dictionnaires ! Alors, envoyez QUE le dictionnaire directement associé au bigramme de mots précédents.
		\item la boucle se termine quand la fonction \textit{sample\_word} détermine que le mot prochain est 'END'.
	\end{itemize}
\end{enumerate}

La fonction \textit{generate()} doit répéter (avec une boucle) la procédure décrite précédemment le nombre de fois indiqué dans l'argument afin de créer le nombre de phrases souhaité.

%\section{Exercice 3 : Générateur}

%\input{ressources/td2_exo3.tex}



\global\long\def\numTD{TD2}%
 
\global\long\def\themeTD{Méthode de Markov appliquée à la classification}%
 
\global\long\def\file{toto.tex}%

\begin{center}
\begin{tabular}{|p{2cm}p{14cm}|}
\hline 
{\includegraphics[width=1.8cm,viewport=0 0 337 248]{img/sorbonne}}  & \raisebox{2ex}{\begin{Large}\textbf{Programmation de Modèles Linguistiques
(II)}\end{Large}}\tabularnewline
2023-2024 & \raisebox{2ex}{(L6SOPROG L3 Sciences du Langage)}\tabularnewline
 & \begin{large}\textbf{TD5}\end{large} \begin{large} \textbf{\themeTD}\end{large} \tabularnewline
 & \tabularnewline
 & Luis MORENO et Caroline Koudoro-Parfait, Sorbonne Université \tabularnewline
\hline 
\end{tabular}
\par\end{center}


\hrule %%%%%%%%%%%%%%%%%%%%%%%%%EN-TETE%%%%%%%%%%%%%%%%%%%%%%%%%%%%%
%\renewcommand{\contentsname}{Sommaire du TD}
%\tableofcontents
%\newpage

\noindent %
\fcolorbox{red}{lightgray}{ %
\begin{minipage}[c]{16.5cm}%
 

\section*{Objectifs}
\begin{itemize}
	\item Mettre en pratique les connaissances acquissent en Python pour l'analyse de corpus linguistiques.
	\item Identifier et extraire les caractéristiques statistiques d'un corpus linguistique.
	\item Construire un modèle linguistique à partir d'une approche stochastique
	\item Concevoir un algorithme capable de recevoir un nouveau texte et calculer le niveau d'appartenance de chaque modèle (classe) entraîné.  
\end{itemize}
%
\end{minipage}} \\

\section{Exercice 1 : Préparation des données}

\vspace{0.5cm}
Pour notre classificateur, nous allons procéder à l’entraînement de deux modèles, un modèle par classe. Chaque classe sera représentée par les œuvres des auteurs : \textit{Edgar Allan Poe} et \textit{Robert Frost}. 

\textbf{MAIS} avant de passer à l'entraînement, nous devons préparer les données afin qu'ils soient traitables pour la machine.

\textbf{Ressources : }

Pour cette implémentation, nous allons avoir besoin de quelques bibliothèques importantes. Considérez que ceci est juste un exemple, si vous avez une autre idée de comment aborder le problème, n’hésitez pas à le faire à votre manière.

Nous allons donc charger les bibliothèques suivantes :

\begin{python}
import numpy as np
import matplotlib.pyplot as plt
import string
from sklearn.model_selection import train_test_split
\end{python}

Si vous travaillez en local, il faut installer les bibliothèques préalablement.

\vspace{0.5cm}

\subsection{Étape 1 : Téléchargement}

La première étape consiste a télécharger les fichiers qui contiennent les textes qui serviront pour entraîner et valider nos modèles. Vous pouvez trouver les fichiers dans la section de Moodle "Ressources".

\subsection{Étape 2 : Lecture de fichiers}

Pour être certains que vous avez bien téléchargé les fichiers et qu'ils sont dans le répertoire principal, testez la commande :

\begin{python} 
!head edgar_allan_poe.txt #Afficher les 10 premières lignes

!head robert_frost.txt
\end{python}

Observez ce qui se passe...


\subsubsection{Enregistrer titres des corpus}

Créez une liste nommée \textit{input\_files}, la liste doit contenir les noms des fichiers que vous venez de télécharger.% Ceci va être très utile plus tard, vous pouvez le faire avec la commande suivante :

%\begin{python}
%input_files = [
%'edgar_allan_poe.txt',
%'robert_frost.txt',
%]
%\end{python}

\subsubsection{Lire les corpus}

\begin{itemize}
	\item Créez deux listes : \textit{input\_texts} et \textit{labels}.
	\item Créez une boucle avec le fonction \textit{enumerate()} en lui donnant comme argument la liste \textit{input\_files}. La fonction \textit{enumerate()} renvoie le contenu de l'objet donné commer argument, ainsi que l'index de chaque élément. Affectez ces valeurs à deux variables : \textit{f} et \textit{label} respectivement.
	\item Dans une deuxième bouclé imbriquée, parcourez le contenu des deux corpus téléchargés, utilisez la fonction \textit{open()} en lui donnant \textit{f} comme argument. Pourquoi \textit{f} ?
	\item Pour chaque itération de la boucle, la fonction \textit{open()} va renvoyer ligne par ligne le contenu du fichier indiqué en argument. Appliquez le prétraitement suivant :
	\begin{itemize}
		\item Convertissez toute la ligne en minuscule;
		\item supprimez le dernier caractère a droit (normalement l'instruction de retour à la ligne) avec la fonction \textit{rstrip()};
		\item supprimez tous les signes de ponctuation
	\end{itemize}
	 et . Affectez le résultat dans une nouvelle variable appelée \textit{line}.
	\item Avec la fonction \textit{append}, rajoutez la ligne pré-traitée à la liste \textit{input\_texts} en parallèle vous devez rajouter aussi l'étiquette associé à cette ligne (\textit{label}) à la liste \textit{labels}.
\end{itemize}

%La première boucle va juste nous montrer chaque élément dans la liste  \textit{input\_files}. On utilise la fonction \textit{enumerate()} car elle retourne les éléments de la liste et sa position. Étant donné que la liste contient les noms des fichiers et que chaque fichier va représenter une classe, on va associer chaque classe avec son index comme son identificateur numérique ou son étiquette.

%La deuxième liste va lire toutes les lignes des fichiers, convertir tout en minuscules et supprimer les ponctuation avant de les enregistrer dans l'objet \textit{input\_texts}. Notez que l'enregistrement se fait ligne par ligne, et de façon parallele, on enregistre aussi l'étiquette correspondante à la ligne dans l'objet \textit{labels.}


%\begin{python}

%# collect data into lists
%input_texts = []
%labels = []

%for label, f in enumerate(input_files):
% print(f"{f} corresponds to label {label}")

% for line in open(f):
%  line = line.rstrip().lower()
%  if line:
%   # remove punctuation
%   line = line.translate(str.maketrans('', '', #\_
%   string.punctuation))

%   input_texts.append(line)
%   labels.append(label)

%\end{python}

\subsection{Étape 3 : Division de corpus}

Nous allons réserver une partie de notre corpus pour l'évaluation. Par norme, on utilise le 70\% pour l'entraînement et le 30\% pour l'évaluation. Vous pouvez utiliser n'importe quelle technique pour effectuer cette séparation, soit directement sur python, ou bien avec des commandes \textit{bash} sur linux. Pour praticité, dans cet exemple nous allons utiliser la fonction \textit{train\_test\_split()} contenue dans la bibliothèque \textit{sklearn} que nous avons importé au début. 

Implémentez la ligne de code suivante :

\begin{python}
train_text, test_text, Ytrain, Ytest = 
train_test_split(input_texts, labels)
\end{python}
Observez que nous envoyons comme arguments les lignes du texte et ses étiquettes, comme ça nous allons garder la trace qui va nous indiquer à quelle classe appartient chaque ligne, même si elles ne sont plus dans le même ordre.

\begin{itemize}
	\item \textbf{train\_text} : contient le texte d’entraînement
	\item \textbf{test\_text} : contient le texte d'évaluation
	\item \textbf{Ytrain} : contient les étiquettes associées aux lignes du texte d'entraînement
	\item \textbf{Ytrain} : contient les étiquettes associées aux lignes du texte d'évaluation
\end{itemize}

Nous allons tester si la fonction a bien effectué la tâche demandée. Essayez le code suivant :
\begin{python}
len(Ytrain), len(Ytest)
train_text[:5]
Ytrain[:5]
\end{python}
Qu'est-ce que les commandes précédentes montrent ?

\subsection{Étape 4 : \textit{mapping}}

Comme vous sûrement savez, la machine n'est pas capable de travailler avec des données linguistiques comme les mots. Nous avons besoin de transformer les mots de notre vocabulaire en valeur numériques exploitables pour la machine.
Nous devons associer alors, chaque mot du vocabulaire à une valeur entière unique. Pour ce faire, nous avons besoin d'un objet où l'on puisse enregistrer une association de genre (${token \rightarrow id}$).
Quel objet peut-on utiliser ? Nommez l'objet \textit{word2idx}\\

Pensez à une stratégie pour parcourir tout le vocabulaire, mot par mot et à chaque nouveau mot affectez \textit{word2idx}. Il doit contenir aussi une représentation pour les mots inconnus, par exemple, les mots qui sont dans le texte d'évaluation mais pas dans le texte d'entraînement, qui nous seront donc inconnu lors de l'étape d'évaluation.

Vous pouvez vous inspirer du morceau de code suivant :

\begin{python}
idx = 1
word2idx = {'<unk>': 0}
for word in vocabulaire:
 if word not known :
  word2idx.add(word, idx)
  idx++
\end{python}

\subsubsection{Stratégie}

\begin{enumerate}
	\item Créez une boucle pour parcourir les éléments de \textit{train\_text} afin d'obtenir une ligne à la fois.
	\item Pour chaque itération faite une découpage de la ligne en mots avec la fonction \textit{split()}, affectez la liste de mot renvoyées par \textit{split()} a une variable (\textit{tokens}).
	\item Dans une deuxième boucle imbriquée, parcourez mot a mot les mots dans \textit{tokens}, validez si le mot courant n'existe déjà dans \textit{word2idx}, s'il n'existe pas, rajoutez le au dictionnaire comme \textit{key} et comme \textit{value} indiquez la valeur actuelle de \textit{idx}.
	\item Ensuite, incrémentez la valeur de \textit{idx} afin d'avoir une valeur unique associée a chaque nouveau mot.
\end{enumerate}


\begin{python}
	if token not in word2idx:
	 word2idx[token] = idx
	 idx += 1
\end{python}
%\begin{python}
%# populate word2idx
% for text in train_text:
%  tokens = text.split()
%   for token in tokens:
%    if token not in word2idx:
%     word2idx[token] = idx
%     idx += 1
%\end{python}

\begin{itemize}
	\item Affichez le contenu du dictionnaire \textit{word2idx} afin de vérifier s'il est bien constitué.
	\item Affichez aussi la taille de notre vocabulaire.
\end{itemize}

\subsection{Étape 6 : Remplacement des mots par leurs représentations numériques}

Maintenant nous allons remplacer les mots dans le vocabulaire d'entraînement par leurs valeurs numériques à l'aide de l'objet \textit{word2idx}. Vous devez créer un nouvel objet pour enregistrer les remplacements effectués.

\begin{python}
	# convert data into integer format
	train_text_int = []
	test_text_int = []
	
\end{python}
\begin{enumerate}
	\item Parcourez ligne par ligne l'objet \textit{train\_text}, utilisez une boucle.
	\item Ensuite, découpez la ligne obtenue en mots, utilisez la fonction \textit{split()}.
	\item À l'aide d'une deuxième boucle imbriquée, parcourez la liste de mots renvoyée par la fonction \textit{split()}.
	\item Dans la deuxième boucle, cherchez la représentation numérique de chaque mot de la phrase analysée.
	\item Enregistrez les représentations trouvées dans une liste temporaire (doit être réinitialisé à chaque itération) dans le même ordre d'apparition des mots.
	\item À la fin de la lecture de chaque phrase, affectez la liste temporaire avec les représentation numériques à l'objet \textit{train\_text\_int}.
	\item \textcolor{red}{L'objet \textit{train\_text\_int} peut être illustré comme une matrice dont chaque ligne est une liste représentant chaque phrase du vocabulaire}.
\end{enumerate}  

%\begin{python}
%for text in train_text:
% tokens = text.split()
% line_as_int = [word2idx[token] for token in tokens]
% train_text_int.append(line_as_int)
%\end{python}

%for text in test_text:
% tokens = text.split()
% line_as_int = [word2idx.get(token, 0) for token in tokens]
% test_text_int.append(line_as_int)

%Notez que dans notre implémentation, nous avons construit une nouvelle liste avec la même architecture que \textit{train\_text} (ligne $\times$ position), en remplaçant les mots par leurs idx.

\textbf{
Répétez la même procédure pour le corpus d'évaluation en affectant la liste \textit{test\_text\_int}}

Regardez les lignes 50, 100, 150, etc de l'objet \textit{train\_text\_int} et comparez les avec les mêmes lignes de l'objet \textit{train\_text}

\section{Exercice 2 : Entraînement du modèle}

Une fois que nous avons les données prêtes, nous pouvons procéder à l'entraînement.

\subsection{Étape 1 : Création des représentations matricielles}
Pour commencer l'entraînement nous devons déclarer les objets qui nous serviront pour traiter les données et effectuer les calculs nécessaires. Rappelez qu'une façon d’illustrer une chaîne de Markov peut être au moyen d'un graphe ou d'une matrice.

Créez deux matrices $A0$ et $A1$ avec une dimensions égal a $V$ où $V = $ \textit{len(word2idx)}. La matrice $A0$ servira pour gérer les probabilités de passer d'un état $t$ étant donné l'état $t-1$ issues du corpus \textit{Poe} et la matrice $A1$ pour le corpus \textit{Frost}. Créez aussi deux listes $pi0$ et $pi1$ pour gérer les probabilités d'apparition des en début de phrase.

Toutes les listes doivent être initialisées en 1. Utilisez la fonction ones() de la bibliothèque \textit{Numpy} comme dans l'exemple.

\begin{python}
#Create a matrix of dimension 10 x 10 initialized by ones
np.ones((10, 10))

#Create a list or vector of 10 initialized by ones
np.ones(10)
\end{python}

\subsection{Étape 2 : Extraction de caractéristiques stochastiques}

Créez la fonction \textit{compute\_counts()}, cette fonction doit recevoir l'objet \textit{text\_as\_int} crée dans l'exercice précédent; une matrice $A$ et une liste $pi$ qui seront affectées avec les caractéristiques issues de \textit{text\_as\_int}.
\begin{python}
# compute counts for A and pi
def compute_counts(text_as_int, A, pi):
\end{python}

À l’intérieure de la fonction lisez \textit{text\_as\_int} pour extraire et comptabiliser les couples ou bigrammes de mots et les affecter dans la matrice $A$. Identifiez et comptabilisez aussi les mots qui apparaissent en début de phrases et gardez le résultat dans la liste $pi$. Pas besoin de \textit{return} dans cette fonction. 

\textcolor{red}{Vous pouvez vous inspirer du code suivant, attention qu'il est pas fonctionnelle; réfléchissez et corrigez le avant de l'ajouter a la fonction.}

\begin{python}
for tokens in text_as_int:

 for idx in tokens:
 if idx is First :
  # it's the first word in a sentence
  pi[idx] += 1
 else:
  # the last word exists, so count a transition
  A[last_idx, idx] += 1

\end{python}

\textbf{Exercice :
Analysez le code suivant et expliquez son fonctionnement.}
\begin{python}
compute_counts([t for t, y in zip(train_text_int, Ytrain)
 if y == 0], A0, pi0)
\end{python}
\vspace{5cm}


\subsection{Étape 3 : Normalisation}

La normalisation va nous permettre d'arranger les valeurs dans l'intervalle [0:1]. Pour cela, nous allons employer une technique classique de distribution de probabilité. Modifiez chaque valeur de la matrice en lui divisant par la somme des valeurs de sa ligne.

Pour corroborer que le processus a été bien effectué, la somme de chaque ligne de la matrice doit être égale à $1$.

\begin{enumerate}
	\item Parcourez ligne à ligne la matrice $A$.
	\item Comptez les valeurs de la ligne et affectez le résultat dans une variable \textit{somme}, utilisez la fonction \textit{sum()}.
	\item Parcourez la ligne et pour chaque élément remplacez sa valeur en lui divisant par \textit{somme}
	\item Répétez le même processus pour chaque ligne.
	\item Répétez le même processus pour chaque classe.
\end{enumerate}
Utilisez la même méthode pour normaliser les valeurs de $pi$.

%\begin{python}
%# normalize A and pi so they are valid probability matrices
%# convince yourself that this is equivalent to the formulas 
%#shown before
%A0 /= A0.sum(axis=1, keepdims=True)
%pi0 /= pi0.sum()

%A1 /= A1.sum(axis=1, keepdims=True)
%pi1 /= pi1.sum()
%\end{python}

\subsection{Étape 4 : propriété \textit{log}}

Pour calculer le logarithme de chaque valeur de nos matrices, nous utilisons la fonction \textit{log()} de la bibliothèque \textit{numpy}.
\begin{python}
# log A and pi since we don't need the actual probs
logA0 = np.log(A0)
logpi0 = np.log(pi0)

logA1 = np.log(A1)
logpi1 = np.log(pi1)
\end{python}

\subsection{Étape 5 : \textit{priors}}

Afin de pouvoir utiliser les règles bayésiennes, nous devons aussi calculer les \textit{priors} de chaque classe. Considérez le code suivant et créez votre propre méthode en utilisant les mêmes noms des objets que dans l'exemple.

\begin{python}
#Nb d'éléments annotés avec label=0
count0 = sum(Ytrain where label==0)
#Nb d'éléments annotés avec label=1
count1 = sum(Ytrain where label==1)
total = len(Ytrain)
p0 = count0 / total
p1 = count1 / total
\end{python}

%\begin{python}
%	# compute priors
%	count0 = sum(y == 0 for y in Ytrain)
%	count1 = sum(y == 1 for y in Ytrain)
%	total = len(Ytrain)
%	p0 = count0 / total
%	p1 = count1 / total
%\end{python}

Et pour être consistant, ne devons aussi calculer les logarithmes.

\begin{python}
logp0 = np.log(p0)
logp1 = np.log(p1)
\end{python}


\section{Exercice 3 : Classificateur}

Maintenant, nous avons tous les éléments pour réaliser une prédiction selon Bayes.

\subsection{Construction de la classe \textit{classifier}}

\begin{enumerate}
	\item Créez la classe \textit{classifier} avec son constructeur, le constructeur doit recevoir comme arguments les listes : \textit{logAs, Logpis, logpriors}. Notez que les noms sont en pluriels, ca veut dire que ces objets vont recevoir des listes de listes, une par classe.
	\begin{itemize}
		\item le constructeur devra initialiser des variables locales en utilisant les valeur reçues en argument.
	\end{itemize}
	\item Créez la fonction \textit{compute\_log\_likelihood()} qui reçoit trois arguments: \textit{input\_, class\_, self}.
	\begin{itemize}
		\item \textit{input\_} : Contiendra le texte à analyser déjà transformé en valeurs entières.
		\item \textit{class\_} : Contiendra l'étiquette de la classe avec la quelle le texte sera comparé (0 = Poe, 1 = Frost)
		\item \textit{self} : Pour accéder aux objets de la classe.
		\item La fonction doit renvoyer la somme des probabilités d'apparition des bigrammes et des mots en début des phrases trouvés dans \textit{input\_}, selon l'équation suivante : $\log p(s_1...T)=\log \pi_{s1}+\sum_{t=2}^{T}\log A_{s_{t-1},s_t}$.
		\item \textbf{Attention !} le calcul doit être effectué selon la classe donnée comme argument.
	\end{itemize}
	\item Créez la fonction \textit{predict()} avec deux arguments comme entrée : \textit{inputs} et \textit{self}.
	\begin{itemize}
		\item \textit{input} : Contiendra les textes à analyser déjà transformés en valeurs entières.
		\item Si l'on part de l'idée que \textit{inputs} contient un poème par ligne
		\begin{itemize}
			\item La fonction doit renvoyer un vecteur de taille = $len(inputs)$.
			\item Chaque position du vecteur doit contenir la classe prédite pour chaque poème donné en argument en respectant l'ordre d'apparition des poèmes.
			\item N'oubliez pas que la classe prédite doit être celle avec la probabilité plus élevée. Rappelez l'équation suivante :\\
			$k^* = \arg max_k \log p(poem | author = k) p(author = k)$
		\end{itemize}
	\end{itemize}
\end{enumerate}
%\exer{Bonus : améliorations}
%%\begin{python}
%%print("texte")
%%\end{python}


\global\long\def\numTD{TD2}%
 
\global\long\def\themeTD{Entraînement des modèles linguistiques}%
 
\global\long\def\file{toto.tex}%

\begin{center}
\begin{tabular}{|p{2cm}p{14cm}|}
\hline 
{\includegraphics[width=1.8cm,viewport=0 0 337 248]{img/sorbonne}}  & \raisebox{2ex}{\begin{Large}\textbf{Programmation de Modèles Linguistiques
(II)}\end{Large}}\tabularnewline
2023-2024 & \raisebox{2ex}{(L6SOPROG L3 Sciences du Langage)}\tabularnewline
 & \begin{large}\textbf{TD5}\end{large} \begin{large} \textbf{\themeTD}\end{large} \tabularnewline
 & \tabularnewline
 & Luis MORENO et Caroline Koudoro-Parfait, Sorbonne Université \tabularnewline
\hline 
\end{tabular}
\par\end{center}


\hrule %%%%%%%%%%%%%%%%%%%%%%%%%EN-TETE%%%%%%%%%%%%%%%%%%%%%%%%%%%%%
%\renewcommand{\contentsname}{Sommaire du TD}
%\tableofcontents
%\newpage

\noindent %
\fcolorbox{red}{lightgray}{ %
\begin{minipage}[c]{16.5cm}%
 

\section*{Objectifs}
\begin{itemize}
\item Construire un modèle linguistique à partir d'une approche stochastique
\item Identifier et extraire les caractéristiques statistiques d'un corpus linguistique. 
\item Mettre en pratique les connaissances acquisse en Python pour l'analyse du corpus
\item Concevoir un algorithme capable de recevoir un nouveau texte et calculer le niveau appartenance de chaque modèle (classe) entraîné.  
\end{itemize}
%
\end{minipage}} \\

\section{Exercice 1 : Préparation des données}

\vspace{0.5cm}
Pour notre petit classificateur, nous allons procéder à l’entraînement de deux modèles, un modèle par classe. Chaque classe sera représentée par les ouvres des auteurs : \textit{Edgar Allan Poe} et \textit{Robert Frost}. 

\textbf{MAIS} avant de passer à l'entraînement, nous devons préparer les données afin qu'ils soient traitables pour la machine.

\vspace{0.5cm}

\textbf{Les étapes :}

\begin{enumerate}
	\item Télécharger les deux corpus
	\item Lire chaque fichier, enregistrer chaque ligne comme un objet de type \textit{list}
	\item Garder les étiquettes(1,2,...,N), une étiquette par classe.
	\item Diviser les corpus (70\% entraînement, 30\% test).
	\item Créer un \textit{mapping} (map == dict) pour associer a chaque mot unique un index numérique.
	\item Créer une boucle pour lire les lignes enregistrées dans l'objet \textit{list}
	\begin{itemize}
		\item Tokenize chaque ligne en mots (utilisez la fonction split())
	\end{itemize} 
	\item Nourrir le dictionnaire (\textit{dict}) en associant chaque mot unique à un index unique numérique.
	\item Créer un index spécial `pour les mots inconnus, des mots qui seront dans le corpus de teste mais pas dans celui d'entraînement.
	\item Changez les mots des lignes du texte par des données numériques.
	\begin{itemize}
		\item employez les index de votre dictionnaire.
		\item Créez des nouvelles listes alignées si nécessaire.
	\end{itemize}
	
\end{enumerate}



%\begin{figure}[h]
%\includegraphics[width=0.6\textwidth]{../images/distrib.png}
%\caption{Distribution des mots par rapport à leur taille en caractères dans
%différentes langues\label{distrib}}
%\end{figure}

\textbf{Etape 1 : Téléchargement}

La première chose à faire consiste a télécharger les fichier qui contiennent les textes qui serviront pour entraîner et valider nos modèles. Vous pouvez trouver les fichiers aux l'adresses suivantes :

\begin{itemize}
	\item \url{https://raw.githubusercontent.com/lazyprogrammer/machine_learning_examples/master/hmm_class/edgar_allan_poe.txt}
	\item \url{https://raw.githubusercontent.com/lazyprogrammer/machine_learning_examples/master/hmm_class/robert_frost.txt}
\end{itemize}

Si vous utilisez un notebook comme \textsc{Jupyter}ou \textsc{Google Collabs}, vous pouvez utiliser le code suivant :
`
\begin{python}
	!wget -nc https://raw.githubusercontent.com/lazyprogrammer/machine_learning_examples/master/hmm_class/edgar_allan_poe.txt
	!wget -nc https://raw.githubusercontent.com/lazyprogrammer/machine_learning_examples/master/hmm_class/robert_frost.txt
\end{python}
 
\textbf{Si ça ne marche pas c'est que} : 
\begin{itemize}
\item Tout n'est pas au bon endroit (\texttt{File not Found}), regardez
dans l'onglet \texttt{files} de \textsc{Jupyter} pour voir où vous
êtes. 
\item ou que on a un problème d'encoding (\texttt{charmap}), il faut ajouter
ecoding ='utf-8' dans le open : open("13846-0.txt", encoding ="utf-8") 
\end{itemize}
Nous allons commencer par le "Discours de la Méthode", si vous avez
conservé le nom d'origine il devrait s'appeler "13846-0.txt".

\begin{python} with open("13846-0.txt") as f: chaine = f.read()
\end{python}

Et on affiche un bout du texte pour vérifier que ça marche :

\begin{python} print(chaine{[}:100{]}) \end{python}

\textbf{Etape 2 : découper}

On va très simplement découper en mots avec la \textbf{méthode} \textit{split}
\begin{python} listemots = chaine.split()\#approximation des occurrences
print("Nombre de mots : %i" %len(liste_mots))
\end{python}

\textbf{Etape 3 : compter}

On va utiliser un \textbf{dictionnaire} (ou tableau associatif) où
l'on va stocker pour chaque longueur en caractères le nombre de mots
qu'on a rencontré. Le fonctionnement est le suivant: 
\begin{itemize}
\item pour chaque mot de la liste de mots, on calcule sa longueur 
\item on vérifie si on a déjà rencontré un mot de cette longueur: 
\begin{itemize}
\item Si c'est le premier mot pour cette longueur on crée une \textbf{clé}
pour cette longueur à laquelle on affecte la \textbf{valeur} 1 
\item Sinon, on \textbf{incrémente} de 1 la valeur existante 
\end{itemize}
\end{itemize}
\begin{python} diclongueurs = {} \#un dictionnaire vide

for mot in listemots: longueur = len(mot)\#la longueur du mot if longueur
not in diclongueurs: \#on a jamais vu cette longueur de mot diclongueurs{[}longueur{]}=1
\# else: \#on a vu cette longueur de mot diclongueurs{[}longueur{]}+=1

print(diclongueurs)\#pour avoir une vue de ce qu'on a fait

\end{python}

NB: si le processus ne vous semble pas clair, ajoutez au début de
la boucle \textit{for} deux lignes (avec l'indentation) pour suivre
le processus pas à pas :

\begin{python} print(diclongueurs) dd=input("Appuyez sur Enter pour
passer a la suite") \end{python}

\textbf{Etape 4: observer}

Un dictionnaire n'est pas une structure de données ordonnée, pour
vérifier que'on trouve des résultats proche de l'attendu, on va afficher
le nombre d'occurences enregistré dans \texttt{dic\_longueurs} pour
toutes les longueurs de 1 à 30 en utilisant \textbf{l'itérateur} \textit{range}.
Dans le \textit{print} on utilise du \textbf{formatage de chaînes
de caractères}\footnote{Voir par exemple \url{https://stackoverflow.com/questions/5082452/string-formatting-vs-format}}.

\begin{python} for toto in range(1, 31):\#de 1 à 30 (31 est exclu)
nbroccurences = diclongueurs{[}toto{]} print("%i : %i"%(toto, nbr_occurences))
\end{python}

Vous verrez que le code plante car on a des longueurs qui ne sont
pas dans le dictionnaire, on va donc améliorer le code de la façon
suivante:

\begin{python} for toto in range(30): if toto in diclongueurs: nbroccurences
= diclongueurs{[}toto{]} print("%i : %i"%(toto, nbr_occurences))
 else: nbroccurences = 0 print("%i : %i"%(toto, nbr_occurences))
\end{python}

\textbf{Etape 5 : représenter}

Et maintenant c'est magique, on va créer une courbe grâce à la librairie
\texttt{matplotlib}. On va importer cette librairie et la renommer
pour que ça soit plus court à écrire. Puis pour avoir les valeurs
à mettre sur la courbe on va lire les valeurs dans l'ordre croissant
pour les ranger dans une liste nommée \textit{liste\_effectifs}. Pyplot
prend entrée un \textbf{vecteur}, une liste de valeurs ordonées.

\begin{python} import matplotlib.pyplot as pyplot \#import avec alias

listeeffectifs = {[}{]} for toto in range(30): if toto in diclongueurs:\#on
a donc vu des mots de cette longueur listeeffectifs.append(diclongueurs{[}toto{]})
else:\#on en n'a pas vu de cette longueur, on ajoute donc un 0 listeeffectifs.append(0)
pyplot.plot(listeeffectifs)\#on "dessine" pyplot.show()\#"on affiche"

\end{python}

\begin{figure}
\centering{}
\includegraphics[width=0.5\textwidth]{../images/TD1_effectifs1.png}
\caption{"Discours de la Méthode" : nombre de mots par longueur (en abscisse),
en ordonnée l'effectif}
\end{figure}

Maintenant si on veut faire le même calcul pour l'autre texte on a
juste à changer le nom du fichier dans l'étape 1 et à relancer toutes
les cellules. Mais si on avait 100 textes à faire ça ne serait pas
très pratique. Nous allons donc voir dans l'exercice suivant comment
améliorer le code.


\section{Exercice 2 : Entraînement des modèles}

Une fois que nous avons les données prêtes, nous pouvons procéder à l'entraînement.

\subsection{Étape 1 : Création des représentations matricielles}
Pour commencer l'entraînement nous devons déclarer les objets qui nous serviront pour traiter les données et effectuer les calculs nécessaires. Rappelez que une façon d’illustrer une chaîne de Markov peut être au moyen d'un graphe ou d'une matrice.

Créez deux matrices $A0$ et $A1$ avec une dimensions égal a $V$ où $V = $ \textit{len(word2idx)}. La matrice $A0$ servira pour gérer les probabilités de passer d'un état $t$ étant donné l'état $t-1$ issues du corpus \textit{Poe} et la matrice $A1$ pour le corpus \textit{Frost}. Créez aussi deux listes $pi0$ et $pi1$ pour gérer les probabilités d'apparition en début de phrase des mots de chaque corpus.

Toutes les liste doivent être initialisées en 1. Utilisez la fonction ones() de la bibliothèque \textit{Numpy} comme dans l'exemple.

\begin{python}
#Create a matrix of dimension 10 x 10 initialized by ones
np.ones((10, 10))

#Create a list or vector of 10 initialized by ones
np.ones(10)
\end{python}

\subsection{Étape 2 : Extraction de caractéristiques stochastiques}

Ensuite créez la fonction \textit{compute\_counts()}, cette fonction doit recevoir l'objet \textit{text\_as\_int} crée dans l'exercice précédent; la matrice $A$ et la la liste $pi$ qui seront affectées avec les caractéristiques issues de \textit{text\_as\_int}.
\begin{python}
# compute counts for A and pi
def compute_counts(text_as_int, A, pi):
\end{python}

À l’intérieure de la fonction lisez \textit{compute\_counts()} pour extraire et comptabiliser les couples ou bigrammes de mots et les affecter dans la matrice $A$. Identifiez et comptabilisez aussi les mots qui apparaissent en début de phrases et gardez le résultat dans la liste $pi$. Pas besoin de \textit{return} dans cette fonction. 

\textcolor{red}{Vous pouvez vous inspirer du code suivant, attention qu'il est pas fonctionnelle; réfléchissez et corrigez le avant de l'ajouter a la fonction.}

\begin{python}
for tokens in text_as_int:
last_idx = None
for idx in tokens:
if last_idx is None:
# it's the first word in a sentence
pi[idx] += 1
else:
# the last word exists, so count a transition
A[last_idx, idx] += 1

# update last idx
last_idx = idx
\end{python}

\textbf{Exercice :
Analysez le code suivant et expliquez son fonctionnement.}
\begin{python}
compute_counts([t for t, y in zip(train_text_int, Ytrain) #cont
 if y == 0], A0, pi0)
compute_counts([t for t, y in zip(train_text_int, Ytrain) #cont
 if y == 1], A1, pi1)
\end{python}
\vspace{5cm}


\subsection{Étape 3 : Normalisation}
Une fois que nous avons compté les occurrences des éléments étudier, nous allons normaliser les valeurs au moyen du calcul d'une moyenne simple.

\begin{python}
# normalize A and pi so they are valid probability matrices
# convince yourself that this is equivalent to the formulas 
#shown before
A0 /= A0.sum(axis=1, keepdims=True)
pi0 /= pi0.sum()

A1 /= A1.sum(axis=1, keepdims=True)
pi1 /= pi1.sum()
\end{python}

\subsection{Étape 4 : propriété \textit{log}}

Pour calculer le logarithme de chaque valeur de nos matrices, nous utilisons la fonction \textit{log()} de la bibliothèque \textit{numpy}.
\begin{python}
# log A and pi since we don't need the actual probs
logA0 = np.log(A0)
logpi0 = np.log(pi0)

logA1 = np.log(A1)
logpi1 = np.log(pi1)
\end{python}

\subsection{Étape 5 : \textit{priors}}

Afin de pouvoir utiliser les règles bayésiennes, nous devons aussi calculer les \textit{priors} de chaque classe. 

\begin{python}
# compute priors
count0 = sum(y == 0 for y in Ytrain)
count1 = sum(y == 1 for y in Ytrain)
total = len(Ytrain)
p0 = count0 / total
p1 = count1 / total
\end{python}
\textbf{Exercice : Décrivez la fonctionnalité de chaque ligne du code ci-dessus}
\vspace{5cm}

Et pour être consistant, ne devons aussi calculer les logarithmes.

\begin{python}
logp0 = np.log(p0)
logp1 = np.log(p1)
\end{python}

\subsection{Étape 5 : Classe \textit{classifier}}
Maintenant, nous avons tous les éléments pour réaliser une prédiction en suivant l’équation REF, nous avons :
\begin{enumerate}
	\item Créez la classe \textit{classifier} avec son constructeur, le constructeur doit recevoir les arguments : \textit{logAs, Logpis, logpriors}
	\begin{itemize}
		\item le constructeur devra initialiser des variables locales en utilisant les valeur reçues en argument.
	\end{itemize}
	\item Créez la fonction \textit{compute\_log\_likelihood()} avec deux arguments comme entrée : \textit{input\_, class\_} ET l'objet \textit{self} car on devra initialiser des variables locales.
	\begin{itemize}
		\item \textit{input\_} : Contiendra le texte à analyser déjà transformé en valeurs entières.
		\item \textit{class\_} : Contiendra l'étiquette de la classe avec la quelle le texte sera comparé (0 = Poe, 1 = Frost)
	\end{itemize}
	\item Créez la fonction \textit{predict()} avec un argument comme entrée : \textit{inputs} ET l'objet \textit{self} car on devra initialiser des variables locales.
	\begin{itemize}
		\item \textit{input} : Contiendra les textes à analyser déjà transformés en valeurs entières.
	\end{itemize}
\end{enumerate}

%\exer{Bonus : améliorations}
%%\begin{python}
%%print("texte")
%%\end{python}


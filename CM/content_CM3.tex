\begin{frame}
  \frametitle{Plan du cours}
\tableofcontents

\end{frame}


\section{Github : qu'est-ce que c'est et à quoi ça sert ?}
\begin{frame}
  \frametitle{C'est quoi Github ?}
\begin{itemize}
\item \ding{81} service web de dépôt et de versionnage de code
\item \ding{81} s'appuie sur logiciel de gestion de versions Git\footnote{\url{https://git-scm.com/}}
\item \ding{81} développé avec Ruby on Rails et Erlang
\item \ding{81} propose des comptes professionnels payants
\item \ding{81} des comptes gratuits pour les projets de logiciels libres

\end{itemize}

\end{frame}


\begin{frame}
  \frametitle{A quoi ça sert ?}
\begin{itemize}
\item \ding{81} hébergement développement de logiciels
\item \ding{81} gestion de développement de logiciels
\item \ding{81} dépôt public de projets libres \& dépôt privé d'entreprises
\item \ding{81} Travaille collaboratif sur un même programme
\end{itemize}
\end{frame}

\section{Github : fonctionnalités et usages}
\begin{frame}
  \frametitle{Les fonctionnalités}
\begin{itemize}
\item \ding{81} contrôle d'accès des collaborateurs
\item \ding{81} fonctionnalités pour la collaboration : système de branches
\item \ding{81} suivi des bugs
\item \ding{81} gestion des tâches 
\item \ding{81} Documentation du projet : un README a rédigé en Markdown\footnote{\url{https://docs.framasoft.org/fr/grav/markdown.html}}
\item \ding{81} Choisir une licence libre\footnote{\url{https://creativecommons.org/licenses/}}
\end{itemize}
\end{frame}

\begin{frame}
  \frametitle{Comment ça marche ?}

\begin{itemize}
\item \ding{81} Il existe un dépôt local (votre machine) et un dépôt distant (le serveur) 
\item \ding{81} Ils sont liés par un fichier caché : .git
\item \ding{81} Vous travaillez en local
\item \ding{81} Vous partagé sur le serveur
\item \ding{81} Vous pouvez récupérer votre travail
\item \textcolor{green}{\ding{81}} Les utilisateurs peuvent cloner vos dépôts publics
\item \ding{81} Les collègues ayant la permission sur vos dépôt publics et privés peuvent : 
\begin{itemize}
\item \textcolor{green}{\ding{52}} récupérer vos dépôts 
\item \textcolor{green}{\ding{52}} commit et push leurs modifications
\end{itemize}
\end{itemize}
\end{frame}

\begin{frame}
  \frametitle{Comment ça marche ?}
\begin{itemize}
\item \ding{80} Toujours vérifier le statut du dépôt == à jour local + serveur ?
\item \textcolor{orange}{\ding{42}} git pull == mise à jour (maj) dépôt local
\item \ding{80} partager sur le serveur
\begin{itemize}
\item \textcolor{orange}{\ding{42}} git commit "décrire la maj" == lancer le partage sur le serveur
\item \textcolor{orange}{\ding{42}} git push == déposer sur le serveur
\end{itemize}
\end{itemize}
\end{frame}

\section{Github desktop : description de l'interface et usage}
\begin{frame}
  \frametitle{créer un compte sur github}

Rendez-vous sur la page : 
\begin{figure}
\caption{\url{https://github.com/.}}
  \includegraphics[width=6cm]{images/github_inscription.png}
  
  \end{figure}
  
\end{frame}

\begin{frame}
  \frametitle{télécharger github desktop}

Rendez-vous sur la page : 
\begin{figure}
\caption{\url{https://desktop.github.com/?ref_cta=download+desktop&ref_loc=installing+github+desktop&ref_page=docs}}
  \includegraphics[width=6cm]{images/github_desktop_connect.png}
  
  \end{figure}

\ding{52} L'interface Desktop est liée à votre compte Github  
  
\end{frame}

\begin{frame}
  \frametitle{Cloner un dépôt existant ...}

Rendez-vous sur la page : 
\begin{figure}
\caption{\url{https://docs.github.com/en/desktop/adding-and-cloning-repositories/cloning-a-repository-from-github-to-github-desktop}}
  \includegraphics[width=6cm]{images/github_desktop_connect2.png}
  \end{figure}
  
\end{frame}

\begin{frame}
  \frametitle{Cloner un dépôt existant ...}

Rendez-vous sur la page : 
\begin{figure}
\caption{\url{https://docs.github.com/en/desktop/adding-and-cloning-repositories/cloning-a-repository-from-github-to-github-desktop}}
  \includegraphics[width=6cm]{images/github_desktop_connect3.png}
  \end{figure}
 \ding{52} ... le dépôt est en local, sur votre machine 
\end{frame}

\begin{frame}
  \frametitle{Nouvelle version du programme cloné : commit}

Rendez-vous sur la page : 
\begin{figure}
\caption{https://docs.github.com/en/desktop/making-changes-in-a-branch/committing-and-reviewing-changes-to-your-project-in-github-desktop}
  \includegraphics[width=6cm]{images/github_desktop_connect4.png}
  \end{figure}

\end{frame}



\begin{frame}
  \frametitle{Nouvelle version du programme cloné : push}

Rendez-vous sur la page : 
\begin{figure}
\caption{\url{https://docs.github.com/en/desktop/making-changes-in-a-branch/pushing-changes-to-github-from-github-desktop}}
  \includegraphics[width=6cm]{images/github_desktop_connect5.png}
  \end{figure}
  
\end{frame}

\begin{frame}
  \frametitle{... push du dépôt sur le serveur}

Rendez-vous sur la page : 
\begin{figure}
\caption{\url{https://docs.github.com/en/desktop/making-changes-in-a-branch/pushing-changes-to-github-from-github-desktop}}
  \includegraphics[width=6cm]{images/github_push_compte.png}
  \end{figure}
  
\end{frame}

\begin{frame}
  \frametitle{Récupérer le travail des collègues : pull request}

Rendez-vous sur la page : 
\begin{figure}
\caption{\url{https://docs.github.com/en/desktop/working-with-your-remote-repository-on-github-or-github-enterprise/viewing-a-pull-request-in-github-desktop}}
  \includegraphics[width=6cm]{images/pull_request.png}
  \end{figure}
  
\end{frame}


\begin{frame}
  \frametitle{Des branches ?}
\ding{81} Créer différentes branches :\\  
  \begin{itemize}
\item \ding{52} peut être utile pour gérer les conflits de version.
\item \ding{52} ça évite à votre collègue d'écraser votre version quand il partage la sienne ou inversement.
\end{itemize}

\ding{81} Mais les branches ça peut être compliqué à gérer. \\
\textcolor{green}{\ding{81}} Au début on peut essayer de travailler sur la même branche : \\
\begin{itemize}
\item \ding{52} Toujours faire un pull avant de push ses propres modifications
\item \textcolor{orange}{\ding{52}} Github prévient en rouge quand il y a un conflit entre la branche locale et la branche serveur !!!
\end{itemize}


\end{frame}

\begin{frame}
  \frametitle{Documentation}

\ding{80} de la documentation, ça peut aider ...\\
\begin{itemize}
\item \ding{229} \url{https://docs.github.com/fr/get-started}
\item \ding{229} \url{https://gist.github.com/Marsgames/2eb2e0321302640efafa4067b483b427}
\end{itemize}


\end{frame}


\begin{frame}
  \frametitle{Généralités}
  \begin{block}{Organisation du semestre}
 \begin{itemize}
   \item CM + TD le Vendredi de 13h00 à 17h00 (C. Koudoro-Parfait \& LG Moreno Jimenez)
   \item Contrôle continu + projet + contrôle terminal
  
  \end{itemize}
  \end{block}
\begin{block}{Ressources}
  \begin{itemize}
  \item \href{http://paris-sorbonne.hosted.exlibrisgroup.com/F?func=find-c&ccl_term=idn=ppn199563403&local_base=MAH01}{TAL et Linguistique Informatique 1}, ISTE Ed. (Mohamed Z. Kurdi) 
  \item Speech and Language Processing (Dan Jurafsky), \url{https://web.stanford.edu/~jurafsky/slp3/} 
  \item Helpdesk : mail ou bureau 206/211 à Serpente (sur RDV)
  \end{itemize}
   \end{block}
\end{frame}

\begin{frame}
  \frametitle{Plan du cours}
\tableofcontents

\end{frame}
\section{Convertir un jupyter notebook en script .py}


\begin{frame}
 \frametitle{Étapes importantes}
\begin{itemize}

\item \textcolor{green}{\ding{51}} Commenter le code non essentiel 

\item \textcolor{green}{\ding{51}} Factoriser le code Jupyter Notebook en fonctions

\item \textcolor{green}{\ding{51}} Créer des scripts Python pour chaque tâches associées

\end{itemize} 
\end{frame}

\begin{frame}
 \frametitle{Étapes importantes}
\begin{itemize}
\item \textcolor{green}{\ding{51}} Commenter le code non essentiel : \# \\
\ding{220} les parties de programme qui ne \textbf{fonctionnent pas} ou \textbf{expérimentatoires}.

\pause

\item \textcolor{green}{\ding{51}} Factoriser le code Jupyter Notebook en fonctions. Observer votre notebook, n'y a t il pas :\\
\ding{220} des parties de code que vous répétez ?\\
\ding{220} des fonctions que vous répétez inutilement.

\pause

\item \textcolor{green}{\ding{51}} Créer des scripts Python pour des tâches associées.\\
\ding{220} Vous pouvez écrire un scripte initiale dans lequel vous appelez un autre script qui va contenir les fonctions que vous souhaitez utiliser
\end{itemize} 
\end{frame}


\begin{frame}
  \frametitle{Enregistrer le fichier au format python - .py}
  Dans la barre de tâche en haut de l'écran jupyter notebook :
  
  
  Aller dans Fichier \ding{219}  Télécharger au format \ding{219}  Python (.py)
  
  \begin{figure}
  \includegraphics[width=10cm]{images/ynpb_convert_py.png}
  \end{figure}
 
\end{frame}


\subsection{Bonnes pratiques d'écriture d'un programme}

\begin{frame}
  \frametitle{Bonne pratique d'écriture d'un programme}
   \begin{figure}
  \includegraphics[width=8cm]{images/spyder_bonnes_pratiques.png}
  \end{figure}
  
\textcolor{green}{\ding{220}} import \\
\textcolor{green}{\ding{220}} fonctions\\

\textcolor{green}{\ding{220}} Main : 
\vspace{-0.3cm}
\begin{itemize}
\item appel des fonctions
\item boucles ...
\end{itemize}

  
\end{frame}

%\begin{frame}
%%  \frametitle{}
%  
%\end{frame}



\section{Spyder un autre environnement }  

\begin{frame} \frametitle{L'Éditeur 1}
  \begin{figure}
  \includegraphics[width=7cm]{images/spyder_editor.png}
  \end{figure}
  
  \ding{220} +sieurs scripts peuvent être ouverts en même temps
\end{frame}

\begin{frame} \frametitle{L'Éditeur 2}

  \begin{figure}
  \includegraphics[width=7cm]{images/spyder_editor2.png}
  \end{figure}
  
  \ding{220} +sieurs scripts peuvent être ouverts côte à côte, ou l'un au dessus de l'autre.\\
  \ding{220} Cliquer sur
  \vspace{-0.3cm} 
  \begin{figure}
  \includegraphics[width=0.5cm]{images/spyder_param.png}
  \end{figure} 
  \vspace{-0.3cm}
  et sélectionner \textit{split horizontally} ou \textit{split vertically}
\end{frame}

\begin{frame}  \frametitle{La console}
  \begin{figure}
  \includegraphics[width=10cm]{images/spyder_console.png}
  \end{figure}
  \ding{220} des lignes de codes peuvent être lancées depuis la console\\
  \ding{220} Les messages d'erreurs s'affichent dans la console \ding{219} Débogage\\
  \ding{220} Carré rouge en haut à droite de la console  == ordinateur calcul
\end{frame}

\begin{frame}
  \frametitle{Situer son script sur sa machine}
  \begin{figure}
  \includegraphics[width=15cm]{images/spyder_chemin.png}
  \end{figure}
  \ding{220} Haut droite, fil d'Ariane : Indique où votre script est rangé sur la machine\\
  \ding{220} Haut gauche : permet de changer de dossier racine, d'explorer l'arborescence de votre machine
\end{frame}

\begin{frame}
  \frametitle{Renseignements sur l'environnement}
  \begin{figure}
  \includegraphics[width=10cm]{images/spyder_infos.png}
  \end{figure}
  \ding{220} conda : base(3.11.5) \ding{219} Version de python utilisé dans l'environnement\\
  \ding{220} Completions: conda(base) \ding{219} Complétion automatique proposée\\
  \ding{220} LSP: Python \ding{219} Language Server Protocol (LSP) \\
  \ding{220} Line 24, Col 1 \ding{219} situation du curseur sur l'éditeur\\
  \ding{220}UTF-8 : Encodage\\
  \ding{220} LF \ding{219} File EOL status, Linux : LF, Windows : CRLF. Problème de fin de ligne par exemple.\\
  \ding{220} RW \ding{219} File permission \footnote{\url{https://www.warp.dev/terminus/linux-file-permissions-explained}}\\
  \ding{220}Mem 17\% \ding{219} Mémoire globale utilisée
  
\end{frame}

\begin{frame}  
\frametitle{L'explorateur de variable}
\begin{figure}
  \includegraphics[width=10cm]{images/spyder_explo_variable.png}
  \end{figure}
  \ding{220} Accédez au nom de variable, type, longueur du contenu de la variable et valeurs
\end{frame}

\begin{frame}
  \frametitle{Sélectionner un fichier}
  \begin{figure}
  \includegraphics[width=9cm]{images/spyder_Files.png}
  \end{figure}
  \vspace{-0.3cm}
  Pour ouvrir un fichier vous pouvez :
  \begin{itemize}
  \item \ding{220} Sélectionner votre script dans l'explo. de fichier de votre machine et le déposer dans l'éditeur
  \item \ding{220} File \ding{219} Open \ding{219} Sélectionner votre script dans l'explo. de fichier affiché 
  \item \ding{220} dans l'explorateur onglet \ding{219} file \ding{219} naviguer \ding{219} double cliquer
  \end{itemize}
  \end{frame}

\begin{frame}
  \frametitle{Historique}
  \begin{figure}
  \includegraphics[width=10cm]{images/spyder_hystory}
	\end{figure} 
	\ding{220} Consulter l'historique des commandes  
  \end{frame}
  
\begin{frame}
  \frametitle{Débogage base}
  \begin{figure}
  \includegraphics[width=10cm]{images/spyder_signal_erreur.png}
	\end{figure} 
	\ding{220} affiche qu'il y a un problème, passer la souris / cliquer dessus pour avoir le détail. 
  \end{frame}
  
\section{Listes, dico, set (et match !)}

\begin{frame}
  \frametitle{Petit point sur les listes}
  \ding{220} Une liste est une structure de données permettant d'accéder
à un objet par son index.\footnote{+ d'infos sur les listes \url{https://gayerie.dev/docs/python/python3/list.html}}\\


\ding{220} Pour illustrer :\\
\ding{80} supposons qu'une liste représente une rue.\\
\ding{80}Chaque habitant vit dans une maison avec un numéro.\\
\ding{80} Ce numéro est l'index de la liste grâce auquel on peut accéder
à l'habitant .


Syntaxe :
\begin{itemize}
\item* liste =[] :définition d'une liste vide
\item* liste =["Annie" , "Paul"] :défnition avec quelques éléments
\item* liste [0] vaut "Annie" , liste[1] vaut "Paul".
\end{itemize}



\end{frame}
\begin{frame}
  \frametitle{Petit point sur les dictionnaires}

\ding{220} Un dictionnaire est une structure de donnée permettant
d'accéder à un objet par une clef.\footnote{+ d'infos sur les dict. \url{https://gayerie.dev/docs/python/python3/dict.html}}\\
\ding{220} Dans un dictionnaire de langue, on utilise les mots
comme clef afin d'accéder à leur définition, qui sont ici
les objets.\\
\ding{220} Dans un dictionnaire, chaque \textbf{clef est unique}.\\
\ding{220} Par contre, \textbf{une même définition} peut correspondre à
\textbf{+sieurs clef}.
Syntaxe :
\begin{itemize}
\item* dico\_vide =$\left\{\right\}$
\item* dico\_age =$\left\{ "Annie" : \textcolor{orange}{20} , "Paul" :18, "Antonia":  \textcolor{orange}{20} \right\}$
\item* dico\_age [ "Annie" ] = 20 .
\end{itemize}
\end{frame}

\begin{frame}
  \frametitle{Petit point sur les set - ensembles}
  \ding{220} c'est un groupement non ordonné d’éléments uniques \ding{220} pas de doublon\footnote{+ d'infos sur les set \url{https://gayerie.dev/docs/python/python3/set.html}}.\\
  \ding{220} peut contenir des éléments de types différents.\\
  \ding{220} pas possible d’accéder aux éléments d’un ensemble avec l’opérateur [] 
  
  Syntaxe :
  
  \begin{itemize}
\item*  mon\_ensemble\_vide = set() !=  mon\_dico =$\left\{\right\}$
\item* mon\_ensemble = $\left\{None, 10, "Bonjour",True \right\}$
\item* 10 in mon\_ensemble \ding{220 } True
\item* 12 in mon\_ensemble \ding{220 } False
\item* 12 not in mon\_ensemble \ding{220 } True
\end{itemize}
\end{frame}

\begin{frame}
  \frametitle{Petit point sur n-uplet ou tuple}
  
 \ding{220} Séquence non modifiable de données ordonnées\footnote{+ d'infos sur les tuple \url{https://gayerie.dev/docs/python/python3/tuple.html}}. \\
 \ding{220} Chaque élément du tuple est associé à une position (un index).\\ 
 \ding{220} Les tuples ne sont pas modifiables.\\ 
 
 Syntaxe :
 
   \begin{itemize}
   
\item*  tuple\_vide = ()
\item* mon\_tuple = (10, 20, 30, 40)
\item*  mon\_tuple = ("caroline",) != chaine=("caroline") 
\item* tuple\_depareille = (None, 10, "Bonjour", True)
 
\end{itemize}
\end{frame}

\begin{frame}
  \frametitle{Rappels}
  \ding{220} Les ensembles, listes et tuples sont des collections d’éléments\\ 
  \ding{220} il est possible de passer de l’un à l’autre grâce aux \textbf{méthode} set(), list(), tuple().\\
  
   \ding{220} un ensemble/set n’est \textbf{pas ordonné} != liste, un tuple ordonnée\\

\ding{220} parcourir un set avec une instruction for, pas de garantie sur l’ordre de parcours des éléments. \\
\ding{220} un set n’est pas une structure de données pertinente quand il existe une relation d’ordre entre les éléments.
\end{frame}









